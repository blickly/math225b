\documentclass[12pt]{article}

%\usepackage{a4wide}
\usepackage[centertags]{amsmath}
\usepackage{amsthm}
\usepackage{amssymb}
\usepackage{amsfonts}
\usepackage{amscd,amsbsy}
%\usepackage{txfonts}
\usepackage[T1]{fontenc} 

% \usepackage{eulervm}
\usepackage{eucal} 
%\usepackage{textcomp}

%\usepackage{mathptmx} 
\usepackage{enumerate}


%\usepackage{pdfsync}


% Deutsche Texte
\usepackage[applemac]{inputenc}

% Seitenlayout
\pagestyle{empty}

\setlength{\textheight}{25cm}
\setlength{\textwidth}{16cm}

\setlength{\hoffset}{-1,5cm}
\setlength{\voffset}{-2,5cm}

\newcommand{\alabel}[1]{\mbox{(#1)}\hfill}
\newenvironment{alist}
        {\begin{list}{}%
                {\renewcommand{\makelabel}{\alabel}%
                \setlength{\parsep}{0.2cm}\setlength{\listparindent}{0cm}
                \setlength{\itemsep}{0.2cm}}}
        {\end{list}}

		\newcounter{aufgabenr}
\newcommand{\Aufg}[1][\mbox{}]{\subsubsection*{Problem \arabic{aufgabenr}#1}\stepcounter{aufgabenr}}
		\stepcounter{aufgabenr}

\newcommand{\problem}[2][\mbox{}]
{\subsubsection*{}
\fbox{\parbox{\textwidth}{
\vspace{-16pt}
\Aufg[#1]
#2}}
\medskip}




% Umgebung "L�sung"
\newenvironment{lsg}{\par\bigskip\noindent{\emph{Solution.\ }}\footnotesize}{\hfill $\blacksquare$}


% Zahlenmengen
\newcommand{\Nat}{\ensuremath{\mathbb{N}}}
\newcommand{\Integer}{\ensuremath{\mathbb{Z}}}
\newcommand{\Rat}{\ensuremath{\mathbb{Q}}}
\newcommand{\Real}{\ensuremath{\mathbb{R}}}
\newcommand{\Comp}{\ensuremath{\mathbb{C}}}
\newcommand{\Conj}[1]{\ensuremath{\overline{#1}}}

\newcommand{\Lang}{\ensuremath{\mathcal{L}}}

\newcommand{\LA}{\ensuremath{\mathcal{L}_A}}
\newcommand{\M}{\ensuremath{\mathcal{M}}}
\newcommand{\TA}{\ensuremath{\operatorname{Th}(\Nat)}}
\newcommand{\PA}{\ensuremath{\mathsf{PA}}}

\newcommand{\Term}[1]{\ensuremath{\underline{#1}}}

\newcommand{\Pow}[1]{\ensuremath{\mathcal{P}(#1)}}




\begin{document}

\begin{center}
{\LARGE Homework 2 for \textbf{MATH 225B}}
%\vspace{0,5cm}

{\large Author: Ben Lickly \\ Due: Thursday Feb 12}
\end{center}

\problem{ Write out primitive recursive derivations (like for $+$ in class) for the following functions: 
$f(x,y) = xy$, $g(x,y) = x^y$, $h(x) = x!$.
}

In class, we defined the primitive recursive definition for addition, or
$p(x,y)$.  The cases here are simple extensions of the same approach.
\begin{eqnarray*}
f'(\vec{x}) &=& p(P_2^3(\vec{x}),P_3^3(\vec{x})) \\
f(0,x) &=& C_0^1(x) = 0 \\
f(y+1, x) &=& f'(y,x,f(y,x)) = x + f(y,x)
\end{eqnarray*}
Thus, $f(y,x) = xy$.
\begin{eqnarray*}
g'(\vec{x}) &=& f(P_2^3(\vec{x}),P_3^3(\vec{x})) \\
g(0,x) &=& C_1^1(x) = 1 \\
g(y+1, x) &=& g'(y,x,g(y,x)) = x \times g(y,x)
\end{eqnarray*}
Thus, $g(y,x) = x^y$.
\begin{eqnarray*}
h'(\vec{x}) &=& f(S(P_1^2(\vec{x})),P_2^2(\vec{x})) \\
h(0) &=& C_1^0() = 1 \\
h(y+1) &=& h'(y,h(y)) = (y+1) \times h(y)
\end{eqnarray*}
Thus, $h(y) = y!$.

\problem{
Show that the functions $(x,y) \mapsto r(x,y)$, $(x,y) \mapsto q(x,y)$, where $r$ and $q$ are such that
\[
	x = q(x,y) \: y + r(x,y), \qquad 0 \leq r(x,y) < y,
\]
are primitive recursive.
}

%divide(x,y)
%  if (x < y):
%    return (0, x)
%  else:
%    return divide(x-y,y) + (1,0)
\begin{eqnarray*}
if(0,x,y) &=& P_1^2(x,y) = x \\
if(n+1,x,y) &=& P_2^2(x,y) = y
\end{eqnarray*}

The following steps use the fact that $|x-y|$ is primitive recursive.
\begin{eqnarray*}
q(0, y) &=& 0 \\
q(x+1, y) &=& q(x,y) + if(|x-y|,1,0)
\end{eqnarray*}
\begin{eqnarray*}
r(0, y) &=& 0 \\
r(x+1, y) &=& if(|x-y|,0,r(x, y) + 1)
\end{eqnarray*}


\problem{
Show that the following functions and relations are primitive recursive:
\begin{enumerate}
\item[a)] $x|y : \iff \exists z \, ( x \cdot z = y)$ \qquad \emph{x divides y}
\item[b)] $D(x)$ \emph{= number of divisors of} $x$
\item[c)] \emph{x is prime}
\item[d)] $p_n$ = \emph{$(n+1)$-st prime} (i.e.\ $p_0 = 2, p_1 = 3, \ldots$)
\item[e)] $(x)_y = 
\begin{cases}
      \text{the exponent of} p_y \text{ in the prime decomposition of } x & \text{if  } x>0, \\
    0  & \text{otherwise}.
\end{cases}$
\end{enumerate}
}

\begin{enumerate}
\item[a)] $\exists z < y \, ( x \cdot z = y)$ is a $\Delta_0$ formula.
\item[b)] 
	This is a bounded search bounded by $x$.
\item[c)] 
	This is a bounded search bounded by $x$.
\item[d)] \label{partd}
	This is a bounded search bounded by ?? %$(n+2)!$.
\item[e)] 
	From part d), we can find $p_y$ with primitive recursion.  The number of triel divisions to determine the exponent of $p_y$ in $x$ can be bounded by $\log x$, so this is also primitive recursive.
\end{enumerate}

\problem{
Write Turing machines which compute the functions $f(x) \equiv k$, $g(x) = 2x$, $h(x,y) = x + y$.
}

The Turing Machine $M_f$ is:
\begin{eqnarray*}
	\{ q_1, 0 \} &\rightarrow& \{ q_2, 1, R \} \\
	\{ q_1, 1 \} &\rightarrow& \{ q_2, 1, R \} \\
	\{ q_2, 0 \} &\rightarrow& \{ q_3, 1, R \} \\
	\{ q_2, 1 \} &\rightarrow& \{ q_3, 1, R \} \\
	 &\ldots& \\
	 \{ q_k, 0 \} &\rightarrow& \{ q_{k+1}, 1, R \} \\
	 \{ q_k, 1 \} &\rightarrow& \{ q_{k+1}, 1, R \} \\
	 \{ q_{k+1}, 0 \} &\rightarrow& \{ q_0, 0, R \} \\
	 \{ q_{k+1}, 1 \} &\rightarrow& \{ q_0, 0, R \} \\
\end{eqnarray*}

The Turing Machine $M_g$ is:
\begin{eqnarray*}
	\{ q_1, 1 \} &\rightarrow& \{ q_1, 0, L \} \\
	\{ q_1, 0 \} &\rightarrow& \{ q_2, 1, R \} \\
	\{ q_2, 0 \} &\rightarrow& \{ q_3, 0, R \} \\
	\{ q_3, 1 \} &\rightarrow& \{ q_3, 1, R \} \\
	\{ q_3, 0 \} &\rightarrow& \{ q_4, 0, L \} \\
	\{ q_4, 1 \} &\rightarrow& \{ q_5, 0, L \} \\
	\{ q_5, 1 \} &\rightarrow& \{ q_5, 5, L \} \\
	\{ q_5, 0 \} &\rightarrow& \{ q_6, 0, L \} \\
	\{ q_6, 1 \} &\rightarrow& \{ q_6, 1, L \} \\
	\{ q_6, 0 \} &\rightarrow& \{ q_7, 1, L \} \\
	\{ q_7, 0 \} &\rightarrow& \{ q_8, 1, R \} \\
	\{ q_8, 1 \} &\rightarrow& \{ q_8, 1, R \} \\
	\{ q_8, 0 \} &\rightarrow& \{ q_3, 0, R \} \\
	\{ q_4, 0 \} &\rightarrow& \{ q_0, 0, L \} \\
\end{eqnarray*}

The Turing Machine $M_h$ is:
\begin{eqnarray*}
	\{ q_1, 1 \} &\rightarrow& \{ q_1, 1, R \} \\
	\{ q_1, 0 \} &\rightarrow& \{ q_2, 1, R \} \\
	\{ q_2, 1 \} &\rightarrow& \{ q_2, 1, R \} \\
	\{ q_2, 0 \} &\rightarrow& \{ q_3, 0, L \} \\
	\{ q_3, 1 \} &\rightarrow& \{ q_4, 0, L \} \\
	\{ q_4, 1 \} &\rightarrow& \{ q_5, 0, L \} \\
	\{ q_5, 1 \} &\rightarrow& \{ q_0, 0, L \} \\
\end{eqnarray*}

\problem{
Consider a Turing machine $M$ with three states (including the halting state $q_0$). We require that $M$ halts on input $0$. What is the maximum number of $1$'s that such an M can output on input $0$?
}

The maximum number is 4, and the associated Turing Machine $M$ is:
\begin{eqnarray*}
	\{ q_1, 0 \} \rightarrow \{ q_2, 1, R \} \\
	\{ q_2, 0 \} \rightarrow \{ q_1, 1, L \} \\
	\{ q_1, 1 \} \rightarrow \{ q_2, 1, L \} \\
	\{ q_2, 1 \} \rightarrow \{ q_0, 1, L \}
\end{eqnarray*}

\end{document}

