\documentclass[12pt]{article}

\usepackage[centertags]{amsmath}
\usepackage{amsthm}
\usepackage{amssymb}
\usepackage{amsfonts}
\usepackage{amscd,amsbsy}
\usepackage{txfonts}
\usepackage[T1]{fontenc} 
\usepackage{fullpage}
\usepackage{ulsy}

\usepackage{eucal} 
\usepackage{enumerate}
\usepackage{pdfsync}
\usepackage{undertilde}

\newcommand{\Nat}{\ensuremath{\mathbb{N}}}
\newcommand{\Integer}{\ensuremath{\mathbb{Z}}}
\newcommand{\Rat}{\ensuremath{\mathbb{Q}}}
\newcommand{\Real}{\ensuremath{\mathbb{R}}}
\newcommand{\Comp}{\ensuremath{\mathbb{C}}}
\newcommand{\Conj}[1]{\ensuremath{\overline{#1}}}

\newcommand{\Lang}{\ensuremath{\mathcal{L}}}

\newcommand{\LA}{\ensuremath{\mathcal{L}_A}}
\newcommand{\M}{\ensuremath{\mathcal{M}}}
\newcommand{\TA}{\ensuremath{\operatorname{Th}(\Nat)}}
\newcommand{\PA}{\ensuremath{\mathsf{PA}}}

\newcommand{\Term}[1]{\ensuremath{\underline{#1}}}

\newcommand{\Pow}[1]{\ensuremath{\mathcal{P}(#1)}}

\newcommand{\defn}{\textbf{Definition}: }
\newcommand{\fixme}{\\ \textbf{FIXME}: Incomplete notes!}

\author{Ben Lickly}
\date{March 10, 2009}
\title{Math 225 Notes}
\begin{document}
\maketitle

\subsection*{COF is $\Sigma_3$-complete}
(i.e. $\Sigma_3 \le_m COF$)

Start with $A \in \Sigma_3$, $x \in A $ iff $\exists y,\ R(x,y)$ (R is $Pi_2$)

\[
 R(x,y) <=> W_{g(x,y)} \text{ infinite}
\]
(g computable)

(At stage $s$,
$
 \Conj{W_{f(x),s}} = \{ b^s_{x,0} < b^s_{x,1} < \ldots < b^s_{x,y} < \ldots \}
$)

Given $x$, enumerate $W_{f(x)}$:
\[
 W_{f(x)} = \bigcup_s W_{f(x),s}
\]                      %^ finite

Stage $s+1$:  $\Conj{W_{f(x),s}} = \{ b^s_{x,0} < b^s_{x,1} < \ldots < b^s_{x,y} < \ldots \}$ \\

For each $y \le s$,
if $W_{g(x,y),s} \ne W_{g(x,y),s+1}$
then enumerate $b^s_{x,y}$ into $W_{f(x),s+1}$


\subsubsection*{Verify this works}

\begin{align*}
 x \in A &=>& \text{ for some } y, W_{g(x,y)} \text{ is infinite} \\
 &=>& \text{ we act infinitely often on behalf of $W_{g(x,y)}$, (new elements show up)} \\
\end{align*}
This means $b^s_{x,y}$ never settles.

Hence: ``The $y$-th element of the complement goes for infinity.'':
\[
 b^s_{x,y} \rightarrow^{s \rightarrow \infty} \infty
\]
$=> \Conj{W_{f(x)}} \le y \Rightarrow f(x) \in COF$

\begin{align*}
 x \not\in A &\Rightarrow& \text{For all $y$, $W_{g(x,y)}$ finite} \\
  &\Rightarrow& \text{For each $y$, we act only finitely often on behalf of $W_{g(x,y)}$} \\
  &\Rightarrow& b^s_{x,y} \text{ settles eventually 
		(if no $W_{g(x,z)}, z \le y$ changes anymore) } \\
  &\Rightarrow& \Conj{W_{f(x)}} = \{ b_{x,y} \}_{y \in \Nat} \quad ( b_{x,y} = \lim_s b^s_{x,y} \\
  &\Rightarrow& W_{f(x)} \text{ infinite} \\
\end{align*}

One can extend this to simultaneous reduction:
\[
 (\Sigma_3, \Pi_3) \le_m (COF, COMP)
\]
$COMP = \{ x : \phi_x \equiv_T K \}$

i.e. for $A \in \Sigma_3$:
\begin{align*}
 f(A) \subseteq COF \\
 f(\Conj{A}) \subseteq COMP
\end{align*}
Since $ COF \subseteq REC = \{ x : W_x \text{ is computable } \}$
 and $REC \cap COMP = \emptyset$.
We get $REC$ is also $\Sigma_3$-complete.

\subsubsection*{Modify proof}
Enumerate $b^s_{x,y}$ into $W_{f(x)}$
if $W_{g(x,y)}$ changes or $y$ gets enumerated into $K$ at step $s$.

\textbf{HW:} Show if $x \not\in A$ then $W_{f(x)} \ge_T K$.

\section*{A char. of the $\Delta^0_2$ T-degrees}

\defn A sequence of (total) functions $\{ f_s \}_{s \in \Nat}$
\emph{converges} (pointwise) to $f$ if f.all $x$,
\[
 f(x) = \lim_s f_s(x)
\]
i.e. if f.all but finitely many $s$, $f(x) = f_s(x)$

\bigskip
$\{f_s\}$ is \emph{computable} if there ex. a comp. function
\[
 \tilde{f} : \Nat \times \Nat \rightarrow \Nat \text{ s.t. }
	f_s(x) = \tilde{f}(x,s)
\]

\bigskip
A \emph{modulus} of convergence for $\{f_s\}$ is a function $m$, s.t.
%                                   ^ assume this converges
For all $s,x$
	if $s \ge m(x)$, 	then $f_s(x) = f(x)$
	(i.e. $f_s(x)$ has settled by time $m(x)$)

$m$ is the \emph{least modulus} if $m(x) = (\mu s) (\forall t \ge s) [f_t(x) = f(x)]$.

Note: If $m$ is a modulus for $\{f_s\}, f_s \rightarrow f$.
%                             ^computable
then $f \le_T m$.
%
b/c $f(x) = f_{m(x)}(x)$

Furthermore, if $g \ge_T m$, then $g \ge_T f$.

\subsection*{Q}
Which functions do have computable approximations? (HW: not $TOT$)

If they have one, how complicated is the modulus?

\section*{Modulus Lemma}
If $A$ is c.e. and $f \le_T A$, then there ex. a computable sequence
$\{ f_s\}$ s.t. $f = \lim_s f_s$ 
and a modulus $m$ of $\{ f_s\}$ s.t. $m \le_T A$.
\begin{proof}
 Sup. $A$ in c.e., $A = W_i$, $f = \Phi^A_e$.
Let $A_s = W_{i,s}$

Def:
\[
 f_s(x) = \begin{cases}
           \Phi^{A_s}_{e,s}(x)	&\text{ if} \Phi^{A_s}_{e,s}(x) \downarrow \\
           0			&\text{ o.w}
          \end{cases}
\]
\begin{align*}
 m(x) = (\mu s)(\exists z \le s) [ \Phi^{A_s \lceil z}_e (x) \downarrow \wedge A_x\lceil z = A \lceil z
\end{align*}
\end{proof}
\begin{itemize}
 \item $\{f_s\}$ is a computable sequence
 \item $m \le_T A$
 \item $m$ is a modulus for $\{f_s\}$, by use principle
\end{itemize}
For non-c.e. sets, we still have comp. approx in case $\le_T \emptyset'$

\newtheorem{limitlemma}{Limit Lemma}
\begin{limitlemma}
 For any function $f$,
$f \le_T A'$ iff there ex. an $A$-comp. sequence
$\{f_s\}$ s.t. $\lim_s f_s = f$.
\end{limitlemma}
\begin{proof}
 (=>) Sup $f \le_T A'$, We know $A'$ is c.e. in $A$.
	=> relativized modulus lemma

 (<=) up $f = \lim_s f_s$, $\{f_s\}$ $A$-comp.
Def: 
\[
 A_x = \{ s : (\exists t) [ s \le t \wedge f_t(x) \ne f_{t+1}(x) ] \}
\] %     ^ stages after which f_s(x) still changes

\begin{itemize}
 \item $A_x$ is finite
 \item $B = \oplus_x A_x = \{ <s,x> : s \in A_x \}$
	Hence $B$ c.e. in $A$
	=> $B \le_T A'$
\end{itemize}

Claim: $f \le_T B \oplus A \le_T A'$

Given $x,s$, using $B$ as an oracle, we can decide whether $f_s(x)$ changes after $s$.

In particular, we can compute the least modulus 
\[
 \mu s (s \not\in A_x)
\]

Hence: \[
A' \ge_T B \oplus A \ge_T m \oplus A \ge_T f
       \]
($B$ computes modulus, $A$ computes approximation)
\end{proof}


In particular: For $A = \emptyset$
\[
 A \le_T \emptyset' \text{ iff $f$ is comp. approximable}
\]

\subsubsection*{Cor}
$F$ has c.e. T-degree iff $f$ is the limit of a comp. seqn. 
$\{f_s\}$ which has a modulus $m \le_T f$.

\begin{proof}
 (=>) $f \equiv_T A$, $A$ c.e.
Modulus lemma => ex. mod $m \le_T A$ for $f$ 

(<=) $f = \lim_s f_s$ with mod $m \le_T A$.

Def: c.e. set $B$ as in pf. of Limit Lemma
then $f \le_T B$

But also $B \le_T m \le_T A \equiv_T f$.

Hence $f \equiv_T B$.
\end{proof}

\section*{Structure of the T-degrees}
\newtheorem{postp}{Post's Problem}
\begin{postp}
 Do there exists incomplete non-computable degrees?

(.i.e does there exist (c.e) $A$ s.t. $\emptyset <_T A <_T \emptyset'$)
\end{postp}

Soultion for non-c.e. (Kleene-Post) [1954]

There exist degrees $\utilde{a}, \utilde{b} \le \utilde{0}$
s.t. $\utilde{a}$ is incomarable with $\utilde{b}$.
(i.e. $\utilde{a} \not\le_T \utilde{b}$ and $\utilde{b} \not\le_T \utilde{a}$
implies $\utilde{0} < \utilde{a},\utilde{b} < \utilde{0}'$

Requirements:
$R_e: A \ne \Phi^B_e$
$S_e: B \ne \Phi^A_e$

Build $A$ and $B$ by extension of finite intial segments.
(like forcing constructions).



\end{document}

