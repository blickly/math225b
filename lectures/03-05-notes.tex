\documentclass[12pt]{article}

\usepackage[centertags]{amsmath}
\usepackage{amsthm}
\usepackage{amssymb}
\usepackage{amsfonts}
\usepackage{amscd,amsbsy}
\usepackage{txfonts}
\usepackage[T1]{fontenc} 
\usepackage{fullpage}
\usepackage{ulsy}

\usepackage{eucal} 
\usepackage{enumerate}
\usepackage{pdfsync}
\usepackage{undertilde}

\newcommand{\Nat}{\ensuremath{\mathbb{N}}}
\newcommand{\Integer}{\ensuremath{\mathbb{Z}}}
\newcommand{\Rat}{\ensuremath{\mathbb{Q}}}
\newcommand{\Real}{\ensuremath{\mathbb{R}}}
\newcommand{\Comp}{\ensuremath{\mathbb{C}}}
\newcommand{\Conj}[1]{\ensuremath{\overline{#1}}}

\newcommand{\Lang}{\ensuremath{\mathcal{L}}}

\newcommand{\LA}{\ensuremath{\mathcal{L}_A}}
\newcommand{\M}{\ensuremath{\mathcal{M}}}
\newcommand{\TA}{\ensuremath{\operatorname{Th}(\Nat)}}
\newcommand{\PA}{\ensuremath{\mathsf{PA}}}

\newcommand{\Term}[1]{\ensuremath{\underline{#1}}}

\newcommand{\Pow}[1]{\ensuremath{\mathcal{P}(#1)}}

\newcommand{\defn}{\textbf{Definition}: }
\newcommand{\fixme}{\\ \textbf{FIXME}: Incomplete notes!}

\author{Ben Lickly}
\date{March 3, 2009}
\title{Math 225 Notes}
\begin{document}
\maketitle

\subsection*{$\Sigma_n \text{ and } \Pi_n$}
\begin{align*}
 Fin &=& \{ x : W_x \text{ finite} \} \\
 &=& \{x : \exists y \forall z (z \ge y \rightarrow z \not\in W_x) \} \\
 &=& \{x : \exists y \forall z \forall s (z \ge y \rightarrow \phi_{x,s}(z) \uparrow) \} \\
\end{align*}
This is $\Sigma_2$
\[
 \Pi_2: Tot
\]
\[
 \Sigma_3: COF, REC
\]

\section*{The Hierarchy Thm}
\defn $A \subseteq \Nat$ is \emph{$\Sigma_n$-complete} (\emph{$\Pi_n$-complete}) if
\begin{enumerate}[(1)]
 \item $A$ is $\Sigma_n$ ($\Pi_n$)
 \item f.all $B \in \Sigma_n (\Pi_N)$, $B \le_1 A$.
\end{enumerate}
Homework: no difference wheter we use $\le_m$ or $\le_1$

Note: $A$ is $\Sigma_n$-computable iff $\Conj{A}$ is $\Pi_n$-complete

To make notation work: Set $\Sigma_0 = \Pi_0 = \Delta_1$.

\newtheorem*{pthm}{Post's Thm}
\begin{pthm} .
\begin{enumerate}[(1)]
 \item 
  \begin{align*}
   B \in \Sigma_{n+1} &\Leftrightarrow& B \text{ is c.e. in some $\Pi_n$ set} \\
   (&\Leftrightarrow& B \text{ is c.e. in some $\Sigma_n$ set} )\\
  \end{align*}
 \item
  $0^{(n)}$ is $\Sigma_n$-complete for $n >0$.
 \item
  $B \in \Sigma_{n+1} \Leftrightarrow B$ c.e. in $0^{(n)}$.
 \item
  $B \in \Delta_{n+1} \Leftrightarrow B \le_T 0^{(n)}$.
\end{enumerate}
\begin{proof}
\begin{enumerate}[(1)]
 \item  (=>) Sup $B \in \Sigma_{n+1}$.
	Then $ x \in B <=> \exists y R(x,y)$ for some $ R \in \Pi_n$ \\
	=> $B$ is $\Sigma_1$ in $R$ \\
	=> $B$ is c.e. in $R$ by relativized Normal FOrm THm for c.e. sets

	(<=) Sup $B$ is c.e. in $C \in \Pi_n$ \\
	Then, for some $e$,
	\begin{align*}
	 x \in B &\Leftrightarrow& x \in W^C_e \\
		&\Leftrightarrow& \exists s \exists \sigma 
			[ \sigma \subset C \wedge x \in W^{\sigma}_{e,s} ] \\
	\sigma \subset C &\Leftrightarrow& \forall y < |\sigma| [ \sigma(y) = C(y) ] \\
		&\Leftrightarrow& \forall y < |\sigma| 
			[ (\sigma(y) = 1 \wedge y \in C) \vee [ (\sigma(y) = 0 \wedge y \not\in C) ] ] \\
	\end{align*}
	Hence, this iis of form ($\Pi_n \cup \Sigma_n$). Now use quant manipulation.
 \item Induction of $n$: \\
	$n = 1$: (halting problem $\qed$) \\
	$n => n+1$: $B \in Sigma_{n+1} \Leftrightarrow B$ c.e. is some $\Sigma_n$ set. \\
	$\Leftrightarrow B$ c.e. in $0^{(n)}$ by IH and Jump Thm \\
	$\Leftrightarrow B \le_1  0^{(n+1)}$ by Jump Thm (3)
 \item by (1) + (2)
 \item 
\begin{align*}
B \text{ is } \Delta_{n+1}  &\Leftrightarrow& B, \Conj{B} \in \Sigma_{n+1} \\
&\Leftrightarrow& B,\Conj{B} \text{ c.e. in } 0^{(n)} \\
&\Leftrightarrow& B \le_T 0^{(n)}, \text{ by relativized Complementation Thm}.
\end{align*}
\end{enumerate}
\end{proof}

Cor: For $n > 0$, $\Delta_n \not\subseteq \Sigma_n$ and $\Delta_n \not\subseteq \Pi_n$.
\begin{proof}
 $0^{(n)} \in \Sigma_n - \Pi_n$ \\
	Ass not: Then $0^{(n)} \in \Pi_n => 0^{(n)} \in \Delta_n 
	=>^{\text{ by (4)}} 0^{(n)} \le_T 0^{(n-1)} ><$

	Likewise $0^{(n)} \in \Pi_n - \Sigma_n$
\end{proof}

\section*{Complete Sets}
THM: $Fin$ is $\Sigma_2$-complete \\
$Tot$ is $\Pi_2$-complete
\begin{proof}
 ``Simultaneous'' reduction
	Fix $ A \in \Sigma_2$. Hence $\Conj{A} \in \Pi_2$
	Tere is a comp. R> s.t.
\[
 x \in \Conj{A} <=> \forall y \exists z R(x,y,z)
\]
Use S-m-n Thm to define comp. function f. s.t. 
\[
 \phi_{f(x)}(n) = \begin{cases}
                   0		&\text{if } (\forall y \le n) (\exists z) R(x,y,z) \\
                   \uparrow	&\text{otherwise}\\
                  \end{cases}
\]
\begin{align*}
 x \in \Conj{A} => \phi_{f(x)}(n) = 0 \forall n => f(x) \in Tot \\
 x \not\in \Conj{A} => x \in A => 
\phi_{f(x)}(n) \text{ defined for only fin. many $n$} => f(x) \in Fin \\
\end{align*}
\end{proof}

THM: $COF$ is $\Sigma_3$-complete.
\begin{proof} ``movable marker construction''.

Let $A$ be $\Sigma_3$  THen, for some $R \in \Pi_2$ \\
$x\in A=> \exists y R(x,y)$

$FIN$ is $\Sigma_2$ complete, hence $INF = \{x : W_x \text{ is infinte} \}$ is
$\Pi_2$ complete.  Therefore, there ex. comp. g. s.t.
\[
 R(x,y) \text{ holds} <=> W_{g(x,y)} \text{ infinite}
\]

Define ce. set $W_{f(x)}$ in stages:
\[
 W_{f(x)} = \bigcup_s W_{f(x),s}
\]
sych that
$x \in A <=> W_{f(x)} \text{ cofinite.}$

(At stage $s$
$
 \Conj{W_{f(x),s}} = \{ b^s_{x,0} < b^s_{x,1} < \ldots < b^s_{x,y} < \ldots \}
$)

Stage 0: $W_{f(x),0} = \emptyset$

Stage $s+1$:  Ass  $\Conj{W_{f(x),s}} = \{ b^s_{x,0} < b^s_{x,1} < \ldots < b^s_{x,y} < \ldots \}$ \\
For each $y \le s$:
If $W_{g(x,y),s} \ne W_{g(x,y),s+1}$
then enumerate $b^s_{x,y}$ into $W_{f(x),s+1}$


\end{proof}



\end{pthm}






\end{document}

