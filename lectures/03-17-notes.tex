\documentclass[12pt]{article}

\usepackage[centertags]{amsmath}
\usepackage{amsthm}
\usepackage{amssymb}
\usepackage{amsfonts}
\usepackage{amscd,amsbsy}
\usepackage{txfonts}
\usepackage[T1]{fontenc} 
\usepackage{fullpage}
\usepackage{ulsy}

\usepackage{eucal} 
\usepackage{enumerate}
\usepackage{pdfsync}
\usepackage{undertilde}

\newcommand{\Nat}{\ensuremath{\mathbb{N}}}
\newcommand{\Integer}{\ensuremath{\mathbb{Z}}}
\newcommand{\Rat}{\ensuremath{\mathbb{Q}}}
\newcommand{\Real}{\ensuremath{\mathbb{R}}}
\newcommand{\Comp}{\ensuremath{\mathbb{C}}}
\newcommand{\Conj}[1]{\ensuremath{\overline{#1}}}

\newcommand{\Lang}{\ensuremath{\mathcal{L}}}

\newcommand{\LA}{\ensuremath{\mathcal{L}_A}}
\newcommand{\M}{\ensuremath{\mathcal{M}}}
\newcommand{\TA}{\ensuremath{\operatorname{Th}(\Nat)}}
\newcommand{\PA}{\ensuremath{\mathsf{PA}}}

\newcommand{\Term}[1]{\ensuremath{\underline{#1}}}

\newcommand{\Pow}[1]{\ensuremath{\mathcal{P}(#1)}}

\newcommand{\defn}{\textbf{Definition}: }
\newcommand{\fixme}{\\ \textbf{FIXME}: Incomplete notes!}

\author{Ben Lickly}
\date{March 17, 2009}
\title{Math 225 Notes}
\begin{document}
\maketitle

\section{Minimal degree theorem}

\newtheorem{mindeg}{Minimal Degree Theorem}
\begin{mindeg}
  There exists a degree $\utilde{a} > \utilde{0}$ s.t. 
  there is no degree $\utilde{b}$ with $\utilde{0} < \utilde{b} < \utilde{a}$
\end{mindeg}
\begin{proof}
  (Idea:) Forcing with perfect computable trees.
\end{proof}

\subsection{Trees
(over $2^{<\Nat}$)}

``simple'' representation:
a tree $T$ is a subset of $2^{<\Nat}$ closed under intial segments

\defn $A \in 2^\Nat$ is an \emph{infinite branch} or \emph{path} through $T$
if f.all $n, A \lceil n \in T$.

$[T]$ denotes the set of infinite branches through $T$.

\textbf{Alternate representation}:
Partial map $T: 2^{<\Nat} \rightarrow 2^{<\Nat}$ s.t.
\begin{align*}
  &\text{(1)}&
     \forall \sigma,\tau 
        \left(T(\sigma)\downarrow \wedge \tau \subseteq \sigma \rightarrow
             (T(\tau)\downarrow \wedge T(\tau) \subseteq T(\sigma)) \right) \\
  &\text{(2)}&
     \forall \sigma \left(T(\sigma \frown 0) \downarrow \rightarrow
        (T(\sigma \frown 0)\downarrow \wedge T(\sigma \frown 0) | %incompatible
                T(\sigma \frown 1)) \right) \\
\end{align*}

Wlog $T$ preserves lexicographical ordering of strings

\defn A tree is \emph{prefect} if $dom(T)$ is $2^{<\Nat}$.
(From now on, we will be dealing only with perfect trees.)

\subsubsection{Notation for trees}

\defn $\tau$ is \emph{on $T$} if, f.some $\sigma$, $T(\sigma) = \tau$.

$\tau$ is compatible with $T$ if there is a 
$\tau^* \supseteq \tau$ that is on $T$.

\defn $T^*$ is a \emph{subtree} of $T$ if $rug(T^*) \subseteq rug(T)$.

Given tree $T$, $\sigma \in 2^{<\Nat}$
\[
 T_\sigma(\tau) = T(\sigma\frown\tau)
\]

\subsection{Forcing notion}
$\mathcal{T}_c$ set of all (computable) perfect trees, 
ordered by the subtree relation.
(i.e. $T \ge T^*$ if $T \supseteq T^*$)

computable: $T$ (as a mapping $2^{<\Nat} \rightarrow 2^{\Nat}$ is computable)

\newtheorem{genex}{Exercise} 
\begin{genex}
  Generic Existence Thm for countable families of dense sets holds.
\end{genex}
(Exercise)

We want to construct sets of natural numbers.
A generic filter , however, is a set of trees.
How do we get from filter to subset of $\Nat$?

Given $G \subseteq \mathcal{T}_c$, define $\mathcal{A}_G \subseteq 2^\Nat$ (i.e subset of $\Pow{\Nat}$)
\[
A \in \mathcal{A}_G: \Leftrightarrow \text{ f.all } T \in G, A \in [T]
\]
\[
\text{ i.e. }
\mathcal{A}_G = \bigcap_{T \in G} [T]
\]

\newtheorem{lem}{Lemma}
\begin{lem}
  Let $\mathcal{C}$ be a collection of dense sets 
  and suppose $G$ is $\mathcal{C}$-generic.
  Then $\mathcal{A}_G \ne \emptyset$
\end{lem}
\begin{proof}
  Inductively construct element $A \in \bigcap_{T \in G} [T]$

  Let $\sigma_0 = \emptyset$.

  Suppose $\sigma_n$ has been defined s.t. f. $i\le n$,
  $\sigma_i$ is compatible with all $T \in G$
  and $\sigma_0 \subset \sigma_1 \subset \dots \subset \sigma_n$.

  Assume there sdoes not exist $k \in \left\{0,1\right\}$ s.t.
  $\sigma_n \frown k$ compatible with all $T \in G$.

  Choose $T_0 \in G$ s.t. $\sigma_n \frown 0$ s not compatible with $T_0$.
  Choose $T_1 \in G$ s.t. $\sigma_n \frown 1$ s not compatible with $T_1$.

  Since $G$ is generic, there exist perfect $T^*$ common
  refinement for $T_0, T_1$.
  This is impossible, since $T^*$ cannot extend $\sigma_n$,
  but every perfect tree has to extend any string compatible with it.
\end{proof}
\subsection{Requirements for minimal degree}

Requirement set $\mathcal{R}$:
\begin{align*}
\text{($A$ not comp) }  
  &R_e:& \chi_A \ne \varphi_e \\
\text{(nothing between 0 and $A$) }  
  &Q_e:& \text{If } \Phi^A_e \text{ is total then either }
        \Phi^A_e \text{ is computable or } A \le_T \Phi^A_e
\end{align*}

\newcommand{\forces}{\Vdash}
\defn $T$ \emph{forces} $R \in \mathcal{R}$ ($T \forces R$), 
if all $A \in [T]$ satisfy $T$.

\newtheorem{densitylem}{Lemma (Density for $R_e$)}
\begin{densitylem}
  Given $T \in \mathcal{T}_c$, there exists subtree $T^*$ of $T$
  s.t. $T^* \forces R_e$.
\end{densitylem}
\begin{proof}
  If $\varphi_e$ is not total, choose $T^* = T$.

  Ass. $\varphi_e$ is total.

  By def. there ex. $x$ s.t.
  \[
      T(0)    (x) + T(1)(x)
  \] % ^string ^xth bit

  Only one of $T(0), T(1)$ can be compatible with $\varphi_e$.

  Choose one which is not, say $i$, and set
  \[
  T^* = T_{\langle i\rangle<++>} % string of lenth 1
  \]
\end{proof}

\newtheorem{densityqe}{Density for $Q_e$}
\begin{densityqe}
  Let $\sigma,\tau,\varrho \in 2^{<\Nat}$

  $(\sigma,\tau)$ is an \emph{e-splitting} of $\varrho$ if
  \[
  \varrho \subset \sigma,\tau
  \]
  and for some $ x \in \Nat$, 
  $\Phi^\sigma_e(x)\downarrow \ne \Phi^\tau_e(x)\downarrow$


  A tree $T$ is \emph{e-splitting} if f.all $\sigma\in 2^{<\Nat}$
  \[
       \left(
        T(\sigma\frown 0), T(\sigma\frown 1)
        \right) \text{ is an e-splitting of $T(\sigma)$}
  \]
\end{densityqe}

\textbf{CRUCIAL FACT}
Sup $T \in \mathcal{T}_c$

Let $A \in [T]$ s.t. $\Phi^A_e$ is  total.

Then
\begin{enumerate}
  \item If there si no e-splitting on $T$,
    then $\Phi^A_e$ is computable.
  \item If $T$ is e-splitting, then $A \le_T \Phi^A_e$. 
\end{enumerate}

\newtheorem{existsdeg}{THM}
\begin{existsdeg}
  There exists a degree $\utilde{a} > 0$ s.t. 
  there is no degree $\utilde{b}$ s.t. $\utilde{0} < \utilde{b} < \utilde{a}$.
\end{existsdeg}

\[
\Phi_e^{T(\sigma\frown 0)}(x)\downarrow \ne 
\Phi_e^{T(\sigma\frown 1)}(x)\downarrow
\]
$\Phi_e^{A}(x)$ agrees with one of them.


\end{document}
