\documentclass[12pt]{article}

\usepackage[centertags]{amsmath}
\usepackage{amsthm}
\usepackage{amssymb}
\usepackage{amsfonts}
\usepackage{amscd,amsbsy}
\usepackage{txfonts}
\usepackage[T1]{fontenc} 
\usepackage{fullpage}
\usepackage{ulsy}

\usepackage{eucal} 
\usepackage{enumerate}
\usepackage{pdfsync}
\usepackage{undertilde}

\newcommand{\Nat}{\ensuremath{\mathbb{N}}}
\newcommand{\Integer}{\ensuremath{\mathbb{Z}}}
\newcommand{\Rat}{\ensuremath{\mathbb{Q}}}
\newcommand{\Real}{\ensuremath{\mathbb{R}}}
\newcommand{\Comp}{\ensuremath{\mathbb{C}}}
\newcommand{\Conj}[1]{\ensuremath{\overline{#1}}}

\newcommand{\Lang}{\ensuremath{\mathcal{L}}}

\newcommand{\LA}{\ensuremath{\mathcal{L}_A}}
\newcommand{\M}{\ensuremath{\mathcal{M}}}
\newcommand{\TA}{\ensuremath{\operatorname{Th}(\Nat)}}
\newcommand{\PA}{\ensuremath{\mathsf{PA}}}

\newcommand{\Term}[1]{\ensuremath{\underline{#1}}}

\newcommand{\Pow}[1]{\ensuremath{\mathcal{P}(#1)}}

\newcommand{\forces}{\Vdash}
\newcommand{\proves}{\vdash}
\newcommand{\gn}[1]{\ulcorner #1 \urcorner}
\newcommand{\defn}{\textbf{Definition}: }
\newcommand{\fixme}{\\ \textbf{FIXME}: Incomplete notes!}

\author{Ben Lickly}
\date{April 7, 2009}
\title{Math 225 Notes}
\begin{document}
\maketitle

\section*{Last Time}
\begin{align*}
  (P_1)& \quad 
  T \proves \sigma \Rightarrow T \proves \Theta_{bwb}(\gn{\sigma}) \\
  (P_2)& \quad
  T \proves \Theta_{bwb}(\gn{\sigma \rightarrow \tau}) \wedge
        \Theta_{bwb}(\gn{\sigma}) \rightarrow \Theta_{bwb}(\gn{\tau}) \\
  (P_3)& \quad
  T \proves \Theta_{bwb}(\gn{\sigma}) \rightarrow 
        \Theta_{bwb}(\gn{\Theta_{bwb}(\gn{\sigma})})
\end{align*}
$(P_1)$ followss directly from reresentability

\section*{This Time}
Let $\Theta_{bew}(x,y)$ be $\Sigma_1$-repres. of $bew_T(a,b) \subseteq \Nat \times \Nat$

\defn \[
\Theta_{bwb}(y) = \exists x, \Theta_{bew}(x,y)
\]
Assume $T \proves \sigma$

Then there ex. $a$ s.t. $bew_T(a, \gn{\sigma})$
\begin{align*}
 \rightarrow& T \proves \Theta_{bew}(\underline{a}, \underline{\gn{\sigma}}) \\
 \rightarrow& T \proves \exists x, \Theta_{bew}(x, \underline{\gn{\sigma}}) \\
 \rightarrow& T \proves  \Theta_{bwb}(\underline{\gn{\sigma}})
\end{align*}

\newtheorem*{blah}{THM}
\begin{blah}
  Let $\delta$ be an operator on $\mathcal{L}_A$-formulas s..t
  \[
  \text{free } \delta \varphi \subseteq \text{free } \varphi
  \]
  s.t.
\begin{align*}
  (d_1)& \quad 
  PA \proves \varphi  \Rightarrow PA \proves \delta\varphi \\
  (d_2)& \quad
  PA \proves \delta(\varphi \rightarrow \psi) \rightarrow
  (\delta\varphi \rightarrow \delta\psi) \\
  (d_3)& \quad
  PA \proves (\delta\varphi)[t/x]  \rightarrow \delta(\varphi[t/x]) \\
\end{align*}
Then 
\[
PA \proves \varphi \rightarrow \delta\varphi 
                \text{ f.all $\Sigma_1$-formulas } \varphi
\]
This implies $(P_3)$ since $\Theta_{bwb}(\gn{\varphi})$ is $\Sigma_1$-sentence.
\end{blah}
\begin{proof}
  works y induction over formulas

  \textbf{Exa:}
  bounded quantification

  Sup $PA \proves \varphi \rightarrow \delta\varphi$
  Let $y \not\in \text{ var } \varphi$
  Let $\psi = (\forall x \subset y) \varphi$
  Show $PA \proves \psi \rightarrow \delta\psi$

  Initial step: $PA \proves \psi[0/y]$
  Hence by $(d_1)$, $PA \proves \delta(\psi[0/y])$
  and by $(d_3)$, $PA \proves (\delta\psi)[0/y]$
  Hence in particular, $PA \proves \psi[0/y] \rightarrow (\delta\psi)[0/y]$

  Inductive Step:
  \begin{align*}
&   PA \proves \psi[y+1 / y] \leftrightarrow (\psi \wedge \varphi[y/x]) \\
&   PA \proves \psi \rightarrow \delta\varphi \text{ yields }
       PA \proves \varphi[y/x] \rightarrow (\delta\varphi)[y/x] \\
\text{Hence } 
&   PA \proves (\psi[y+1 / y] \wedge (\psi \rightarrow \delta\psi) )
       \rightarrow (\psi \wedge \varphi[y/x] \wedge 
                                (\psi \rightarrow \delta\psi)) \\
  \rightarrow
&   PA \proves (\psi[y+1 / y] \wedge (\psi \rightarrow \delta\psi) )
       \rightarrow (\delta\psi \wedge \delta\varphi[y/x]) \\
  \rightarrow
&   PA \proves (\psi[y+1 / y] \wedge (\psi \rightarrow \delta\psi) )
       \rightarrow (\delta\psi[y+1 / y]) \\
  \rightarrow
&   PA \proves (\psi \rightarrow \delta\psi) \rightarrow
            (\psi[y+1 / y] \rightarrow \delta\psi[y+1 / y])
  \end{align*}
\end{proof}

\subsection*{``Philosophical'' Implications}
\[
PA \not\proves CON_{PA}
\]

\[
  PA^\bot = PA + \neg CON_{PA}
\]
Since $PA \not\proves CON_{PA}$, $PA^\bot$ is consistent.

The proof of the 2nd Incompleteness Thm actually yields
\begin{align}
  PA \proves CON_{PA} \leftrightarrow \neg \Theta_{bwb}(\gn{CON_{PA}})
  \label{eq:star}
\end{align}

Additional property of $\Theta_{bwb}$:
%
If $T$ satisfies $(P_1)-(P_2)$
and $T' = T + \sigma$, then
\[
T \proves \Theta_{T'}(\gn{\tau}) \leftrightarrow \Theta_T(\gn{\sigma \rightarrow \tau})
\]
and $(P_1)-(P_2)$ hold for $T'$

Hence 
\begin{align*}
  &PA \proves \Theta_{PA+\sigma}(\gn{0=1}) \leftrightarrow \Theta_{PA}(\gn{\sigma \rightarrow 0 = 1}) \\
 \rightarrow
  &PA \proves \Theta_{PA+\sigma}(\gn{0=1}) \leftrightarrow \Theta_{PA}(\gn{\neg \sigma}) \\
 \rightarrow
  &PA \proves \neg \Theta_{PA+\sigma}(\gn{0=1}) \leftrightarrow \neg \Theta_{PA}(\gn{\neg \sigma}) \\
 \rightarrow
 &PA \proves \neg CON_{PA} \leftrightarrow \neg \Theta_{PA}(\gn{CON_{PA}})
\end{align*}
and thus by equation \ref{eq:star},
\[
PA \proves CON_{PA^\bot} \leftrightarrow CON_{PA}
\]

Hence also \[
 PA^\bot \proves \dots
\]

Since $PA^\bot \proves \neg CON_{PA}$,
it follows that $PA^\bot \proves \neg CON_{PA^\bot}$

\textbf{HW Exercise}: What the heck does this mean???

\end{document}
