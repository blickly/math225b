\documentclass[12pt]{article}

\usepackage[centertags]{amsmath}
\usepackage{amsthm}
\usepackage{amssymb}
\usepackage{amsfonts}
\usepackage{amscd,amsbsy}
\usepackage{txfonts}
\usepackage[T1]{fontenc}
\usepackage{fullpage}
\usepackage{ulsy}

\usepackage{eucal}
\usepackage{enumerate}
\usepackage{pdfsync}
\usepackage{undertilde}

\newcommand{\Nat}{\ensuremath{\mathbb{N}}}
\newcommand{\Integer}{\ensuremath{\mathbb{Z}}}
\newcommand{\Rat}{\ensuremath{\mathbb{Q}}}
\newcommand{\Real}{\ensuremath{\mathbb{R}}}
\newcommand{\Comp}{\ensuremath{\mathbb{C}}}
\newcommand{\Conj}[1]{\ensuremath{\overline{#1}}}

\newcommand{\Lang}{\ensuremath{\mathcal{L}}}

\newcommand{\LA}{\ensuremath{\mathcal{L}_A}}
\newcommand{\M}{\ensuremath{\mathcal{M}}}
\newcommand{\TA}{\ensuremath{\operatorname{Th}(\Nat)}}
\newcommand{\PA}{\ensuremath{\mathsf{PA}}}

\newcommand{\Term}[1]{\ensuremath{\underline{#1}}}

\newcommand{\Pow}[1]{\ensuremath{\mathcal{P}(#1)}}

\newcommand{\forces}{\Vdash}
\newcommand{\proves}{\vdash}
\newcommand{\gn}[1]{\ulcorner #1 \urcorner}
\newcommand{\defn}{\textbf{Definition}: }
\newcommand{\fixme}{\\ \textbf{FIXME}: Incomplete notes!}

\author{Ben Lickly}
\date{April 23, 2009}
\title{Math 225B Notes}
\begin{document}
\maketitle

\section{Last Time}
\newtheorem{parhar}{PaHa Principle}
\begin{parhar}
%{Paris-Harrington Principle}
 For $n, m, r \in \Nat$, there ex. $l \in \Nat$ s.t.

If $[l]^n = C_1 \cup \dots \cup C_r$
there there exists $\underline{Y} \subseteq \left\{ 0, \dots, l-1 \right\}$ s.t.
\begin{itemize}
  \item $\underline{Y}$ is homogeneous
  \item $|\underline{Y}| \ge m$
  \item $|\underline{Y}| \ge$ min $\underline{Y}$ (new)
    ($\underline{Y}$ is ``relatively large'')
\end{itemize}
\end{parhar}

(*) For all $c,m,n,k$ there is a $d$ s.t.
$f_1,\dots,f_k : [d]^n \rightarrow d$ are oregressive
then there is $\underline{Y} \subseteq [c,d)$
\begin{itemize}
  \item $|Y| \ge m$
  \item $\underline{Y}$ min-homogeneous for each $f_i$
\end{itemize}

\subsubsection{Diag-indiscernibles}
$\mathcal{M} \models \PA$, $I \subseteq M$ sequence of
indiscernibles for a fomula $\varphi$
if whenever $z, x_1, \dots, x_n, \dots, y_1, \dots, y_n \in I$ s.t.

$z < x_1 < \dots < x_n$
and
$z < y_1 < \dots < y_n$
and
$a_1, .., a_m < z$

then $\mathcal{M} \models \varphi[\vec{a}, x_1, \dots, x_n] \Leftrightarrow
\mathcal{M} \models \varphi[\vec{a}, y_1, \dots, y_n]$

\subsubsection{Indiscernibles from (*)}
For any $c, l, m, n, k$ and any $\varphi_1, \dots, \varphi_e$
$\mathcal{L}_A$-formulas in $k+n$ variables there is a set $I$ of
diag. indisc. for $\varphi_1, \dots, \varphi_e$ with $|I| \ge m$ and
$min I > c$

\newtheorem*{modelsi}{Models from indiscernibles}
\begin{modelsi}
  Sup $\mathcal{M} \models \PA$, $x_0 < x_1 < \dots$ sequence of diag. indisc.
  for all $\Delta_0$-formulas.

  Let $N = \left\{ y \in \mathcal{M} : y < x_i \text{ for some } i < \omega \right\}$

  THen $N$ is closed under addition and multiplication
  and $\mathcal{N} = (N, +, *, 0, 1) \le \mathcal{M}$ is a model of $\PA$
\end{modelsi}
\begin{proof}
\begin{enumerate}[(1)]
  \item $N$ closed under addition:
    Let $a < x_i$, $b < x_j$, wlog $i < j$

    Then $a + b < x_i + x_j$

    Supp. $x_i + x_j > x_k$ for $k > j$

    Then there ex. $c < x_i$ s.t. $c + x_j = x_k$

    By indiscernability, $c+x_j = x_e$ f.any $e > k$

    \fixme %FIXME

  \item $N$ is closed under multiplication.
    Let $a < x_i$, $b < x_j$, wlog $i < j$

    Ass. $x_i * x_j > x_k$ for $k > j > i$

    Then there ex. $c < e_i$ st..
    (+) $c * x_j < x_k \le (c+1) * x_j$

    add $x_j$:
    \[
(c+1) x_j < x_k + x_j
    \]

    By indiscerability, (+) also implies
    \[
    x_e < (c+1)* x_j \text{  f.all } l > k
    \]
    By proof of closure under $+$,
    \begin{align*}
x_k + x_j < x_e
\rightarrow (c +1) * x_J < x_k + x_j < x_e \le (c+a) * x_j
    \end{align*}
\end{enumerate}
Remains: Induction holds

We will prove an equivalent principle:

Least number principle (LNP):
\[
\forall \vec{y} \left( \exists x \varphi(x, \vec{y}) \rightarrow
\exists z (\varphi(z, \vec{y}) \wedge \forall w < z \neg \varphi(w, \vec{y}))\right)
\]

HW Exercise:
Oer $\PA^-$ LNP is equivalent to induction.
Hence $\mathcal{M} \models \PA^- + LNP$ then $\mathcal{M} \models \PA$
\end{proof}

\subsubsection*{Verify LNP for $\mathcal{N}$:}
Sup $\varphi$ is $\mathcal{L}_A$-formula, and $\vec{a}, b \in N$ s.t.
\[
\mathcal{N} \models \varphi[b, \vec{a}]
\]
How do we find a least $c \in N$ s.t. $\mathcal{N} \models \varphi[c, \vec{a}]$?
(Remember, $\mathcal{M}$ is nonstandard)

\textbf{We know:}
$\mathcal{M}$ satisfies LNP, hence in $\mathcal{M}$ we can find least witness

Problem: $\exists x \varphi$ might not hold in $\mathcal{M}$

Recall: Absoluteness for initial segments:
%
We proved:
$\Delta_0$-formulas are absolute between $\mathcal{L}_A$-structures $\mathcal{N}, \mathcal{M}$ s.t. $\mathcal{N} \subseteq_e \mathcal{M}$ (end-extension).
%
If $\varphi$ is $\Delta_0$ and $\vec{a} \in N$ then
\[
\mathcal{N} \models \varphi[\vec{a}] \text{ iff } \mathcal{M} \models \varphi[\vec{a}]
\]

---Notation:
Write $\mathcal{M} \models \exists x \varphi(x, \vec{a})$
meaning. There ex. $b \in \mathcal{M}$ s.t. $\mathcal{M} \models \varphi[b, \vec{a}]$

If $\varphi$ were $\Delta_0$, we would be fine, by absoluteness
---

\textbf{Trick:} Link truth of $\varphi$ in $\mathcal{N}$ to truth of a
$\Delta_0$-formula in $\mathcal{M}$
\[
\mathcal{N} \models \varphi(b, \vec{a})
\]
$\varphi$ is equivalent to formula $\exists v_1 \forall v_2 \dots Q v_n
\psi(x, \vec{y}, v_1, \dots, v_n)$

Hence
\[
\text{(a)} \mathcal{N} \models \exists v_1 \forall v_2 \dots Q v_n
\psi(b, \vec{a}, v_1, \dots, v_n)
\]

$I$ unbounded in $N$.
Thus for any formula $\Theta$,
\[
\mathcal{N} \models \exists x \Theta(x) \text{ iff exists $i$ s.t.}
\mathcal{N} \models \exists x < x_i \Theta(x)
\]
\[
\mathcal{N} \models \forall x \Theta(x) \text{ iff f.all $i$ s.t.}
\mathcal{N} \models \forall x < x_i \Theta(x)
\]


Let $i$ be s.t. $b, \vec{a} < x_i$
Continuing this argument inductively,
we get
\begin{align*}
(a) \Leftrightarrow \exists i_1 > i, \forall i_2 > i_1, \dots Q i_n > i_{n-1}
\text{ s.t. }
\fixme
  \\
  \Leftrightarrow^{\Delta_0-\text{absoluteness}} \exists i_1 > i, \forall i_2 > i_1, \dots
  \mathcal{M} \models \dots \\
%
(b) \qquad \Leftrightarrow^{I \text{ indisc for } \mathcal{M}}
\mathcal{M} \models
\exists v_i < x_{i+1}, \forall v_2 < x_{i+2}, \dots Q v_n < x_{i+n}
\psi(b, \vec{a}, \vec{v})
\end{align*}

LNP holds in $\mathcal{M}$, hence there ex. least $c < x_i$ s.t.
\[
\mathcal{M} \models \dots \psi(c, \vec{a}, \vec{v})
\]
By def of $N$, $c \in N$ and by (b) and absoluteness,
$\mathcal{N} \models \varphi(c, \vec{a})$ and $c$ is least witness in $\mathcal{N}$

Final argument: If $\mathcal{M} \models \PA + (*)$
then $\mathcal{N}$ is a model in which (*) fails.

\subsection{
Coding Syntax and Semantics in $\PA$
}

By C-T Thesis
\[
S = \left\{ ( \gn{\Theta(\vec{v})}) :
\Theta(v_1, \dots, v_n) \text{ is } \Sigma_1 \text{ and }
\Nat \models \Theta( (a)_1, \dots, (a)_n)\right\}
\]
is c.e.

By Representation Thm, there is a $\Sigma_1$-formula $SAT_{\Sigma_1}(x,y)$
repres. $S$ over $\Nat$

Verify:
$SAT_{\Sigma_1}(x,y)$ has all ``desired'' properties of a truth predicate for
$\Sigma_1$-formulas over $\PA$.
In particular
\[
\PA \proves \forall \vec{x} \left( \varphi(\vec{x}) \leftrightarrow
SAT_{\Sigma_1}(\gn{\varphi}, \vec{x})
\fixme
\right)
\]
\fixme

\begin{itemize}
  \item relation $e(u,v,i)$
    $u$ codes seqn of length $\ge i$ and $v$ is $i$th element of sequence
  \item set of $GN's$
\end{itemize}
Can pick the coding for set s.t. code of a subset of $\left\{ 0, \dots, a-1 \right\}$ is $\le 2$

---

Show $PA \not \proves (*)$

Let $\mathcal{M} \models \PA$ nonstandared s.t. $\mathcal{M} \models (*)$
pick any non-standard $c \in M$

One can show $\PA \proves \text{ Finite R.T}$
$ \rightarrow $ there ex. least $w \in M$ s.t.
$\mathcal{M} \models W \rightarrow \left( 3c + 1 \right)^{2c+1}_c$

Since $\mathcal{M} \models (*)$, there ex. a least $d \in M$ s.t.
(*)-statement holds (in $\mathcal{M}$) with $\underline{Y} \subseteq [c,d)$
$|\underline{Y}| \ge w$
(Formalize (*) in $\mathcal{L}_A$)

``Follow'' proof of ``Indicerbiles from (*)'' inside $\mathcal{M}$
                                % ^
\begin{itemize}                 % ^
 \item Formalize the statement of Lemma
   \[
   \forall c,k,l,m,n,p (p \text{ codes $l$ formula with $k + n$ free var}
   \rightarrow \exists e (\text{ $e$ code for a set}))
   \]
   s.t. $|I| \ge c, min I > c$
   and $I$ is a set of diag indisc formulas in $p$.
   % use SAT pred.

 \item  Note thta LEmma can be proved in $\PA + (*)$
   \[
   \mathcal{M} \models \sigma
   \]
\end{itemize}



\end{document}
