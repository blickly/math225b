\documentclass[12pt]{article}

\usepackage[centertags]{amsmath}
\usepackage{amsthm}
\usepackage{amssymb}
\usepackage{amsfonts}
\usepackage{amscd,amsbsy}
\usepackage{txfonts}
\usepackage[T1]{fontenc} 
\usepackage{fullpage}
\usepackage{ulsy}

\usepackage{eucal} 
\usepackage{enumerate}
\usepackage{pdfsync}
\usepackage{undertilde}

\newcommand{\Nat}{\ensuremath{\mathbb{N}}}
\newcommand{\Integer}{\ensuremath{\mathbb{Z}}}
\newcommand{\Rat}{\ensuremath{\mathbb{Q}}}
\newcommand{\Real}{\ensuremath{\mathbb{R}}}
\newcommand{\Comp}{\ensuremath{\mathbb{C}}}
\newcommand{\Conj}[1]{\ensuremath{\overline{#1}}}

\newcommand{\Lang}{\ensuremath{\mathcal{L}}}

\newcommand{\LA}{\ensuremath{\mathcal{L}_A}}
\newcommand{\M}{\ensuremath{\mathcal{M}}}
\newcommand{\TA}{\ensuremath{\operatorname{Th}(\Nat)}}
\newcommand{\PA}{\ensuremath{\mathsf{PA}}}

\newcommand{\Term}[1]{\ensuremath{\underline{#1}}}

\newcommand{\Pow}[1]{\ensuremath{\mathcal{P}(#1)}}

\newcommand{\defn}{\textbf{Definition}: }
\newcommand{\fixme}{\\ \textbf{FIXME}: Incomplete notes!}

\author{Ben Lickly}
\date{March 3, 2009}
\title{Math 225 Notes}
\begin{document}
\maketitle

\subsection*{Definitions}
Oracle TMs:
$\Phi^A_e$ and $W^A_e$

Turing Reducability
($A \le_T B$):
$A = \Phi^B_e$

\subsection*{}
$A$ is \textit{computable enumerable} in $B$ if $A = W^B_e$ for some $e$.

\section*{Relativized Complementation Theorem}
 $A \le_T B$ iff $A$ and $\Conj{A}$ are c.e. in $B$.

Thm: The following are equivalent:
\begin{enumerate}[(i)]
\item $A$ c.e. in $B$
\item $A = \emptyset$ or $A$ is the range of a total $B$-computable function
\item $A$ is $\Sigma_1^B$
\end{enumerate}

Relativized $\Sigma_1^B$: \\
	A is in $\Sigma_1^B$-form if $A = \{ x : (\exists \vec{y}) R^B(x, \vec{y}) \}$
	where $R^B$ is a relation computable in $B$.

(For definability over $\Nat$: work over second-order arithmetic.)
\[ (\Nat, \Pow{\Nat},    \epsilon, +, *, 0, 1) \]
%  ^two sorted structure ^relation between number and set variables 

 $x 	\in 	B$
%^number var	^ set var.
introduces def'bility from parameters from $\Pow{\Nat}$

\begin{proof}
 relativze the previous proofs

e.g. (i) => (iii) \\
Sup $A = W^B_e$ \\
By Use Principle: 
\[
 x \in A \Leftrightarrow (\exists s) (\exists \sigma) [ \sigma \subset B \wedge W_{e,s}^\sigma]
\]
\end{proof}

\section*{Turing Degrees}

\defn $A \equiv_T B$ iff $A \le_T B$ and $B \le_T A$.

Turing degree of $A$:	$deg_T(A) = \{ B : B \equiv_T A \}$

$\mathcal{D}$ class of T-degrees.\\	$2^{\aleph_0}$ many. \\
parially ordered: $\utilde{a} \le \utilde{b}$ % should be undertilde

each degree is infinite: Given $A$, let $A_n$ 
be given by $ x \in A_n \Leftrightarrow x + n \in A$
iff $A = $ \fixme %FIXME:

\newcommand{\join}{\vee}
\defn upper semilattice: joins exist
\[
 \utilde{a} \join \utilde{b} = deg_T(A \oplus B)
\]
If $\utilde{c} \ge \utilde{a}, \utilde{b}$, then 
$\utilde{c} \ge \utilde{a} \join \utilde{b}$

$\utilde{a} \in \mathcal{D}$ is called c.e. if it contains a c.e. set \\
$\mathcal{R} \subseteq \mathcal{D}$ denotes the set of c.e. degrees. \\
$\utilde{a}$ is c.e. in $\utilde{b}$ if it contains $A \in \utilde{a}$ c.e. in $B \in \utilde{b}$.

\subsection*{The Jump Operator}
\defn \emph{jump} of $A$, denoted by $A'$
\[
 A' = K^A = \{ x : \Phi^A_x(x) \}
\]
For $n \in \Nat$, $A^{(0)} = A$, $A^{(n+1)} = (A^{(n)})'$.

Relativized Basic Thms:
\[
 K^A = K^A_0 = K^A_1 = \{ x : W^A_x \ne \emptyset \}
\]

\newtheorem*{jmpthm}{Jump Theorem}
\begin{jmpthm}
 .
 \begin{enumerate}[(1)]
  \item $A'$ is c.e. in $A$
  \item $A' \not\le_T A$
\begin{proof}
 Ass $A'$ is computable in $A$, say $A' = \Phi^A_e$
 Def. \[
       f(x) = \begin{cases}
               \Phi^A_x(x)+1	&\text{if } Phi^A_e(x) = 1 \\ % i.e. x \in A 
		1 		&\text{if } Phi^A_e(x) = 0 \\ % ie.. x notin A
              \end{cases}
       \]
 % contradiction
\begin{align*}
 d \in A => Phi^A_d(d) \downarrow = f(d) = Phi^A_d(d) +1 \Rightarrow\Leftarrow \\ 
 d \not\in A => Phi^A_d(d) \uparrow = f(d) \Rightarrow\Leftarrow \\
\end{align*}			% f is total
\end{proof}
  \item $B$ is c.e. in $A$ iff $B \le_1 A'$
  \item $A$ is c.e. in $B$, $B \le_T C$ => $A$ c.e. in $C$.
\begin{proof}
 $A \ne \emptyset => A$ is the range of a $B$-comp function. \\
=> $A$ is the range of a $C$-comp function
\fixme % FIXME: Add proof
\end{proof}
  \item $B \le_T A$ iff $B' \le_1 A'$
\begin{proof}
 (=>) $B \le_T A$ => $B'$ is c.e. in $A$ by (4) since $B'$ is c.e. in $B$ and (4).
now apply (3)

 (<=) $B' \le_1 A'$ => $B, \Conj{B}$ c.e. in $A$ by (3) + (4) \\
  => $B \le_T A$ by Compl. Thm.
\end{proof}
  \item $B \equiv_T A$ => $B' \equiv_1 A'$	(in part $B' \equiv_T A'$ )
  \item $A$ c.e. in $B$ iff $A$ c.e. in $\Conj{B}$ 
	\proof follows from (4) since $\Conj{B} \equiv_T B$.


 \end{enumerate}
\end{jmpthm}

COR: Write $\utilde{0}^{(n)} = deg_T( 0^{(n)})$

Thm: $\utilde{0} < \utilde{0}' < \utilde{0}'' < \ldots < \utilde{0}^{(n)} < \ldots$.

\section*{The Arithmetical Hierarchy}

Representation Thm for c.e. sets:
\begin{align*}
 A \text{ c.e.} &<=>& A \text{ proj of comp rel } Z \\
&<=>& A \text{ is  } \Sigma_1-\text{def'able over } \Nat
\end{align*}
Goal: Extend this to more complicated formulas and relate it to the jump hierarchy

\defn $A$ is $\Sigma_n$, $n > 1$, iff there is
	a comp. re. $R(x, y_1, \ldots, y_n)$, s.t
\[
 x \in A \Leftrightarrow \exists y_1 \forall y_2 \exists y_3 \ldots Q y_n, R(x, y_1, \ldots y_n)
\]

\defn $A$ is $\Pi_n$, $n > 1$, iff there is
	a comp. re. $R(x, y_1, \ldots, y_n)$, s.t
\[
 x \in A \Leftrightarrow \forall y_1 \exists y_2 \ldots Q y_n, R(x, y_1, \ldots y_n)
\]

\defn $A$ is $\Delta_n$ if it is $\Sigma_n$ and $\Pi_n$.

\defn $A$ is arithmetical if it is $\Sigma_n$ or $\Pi_n$ for some $n$.

Quantifier manipulation:
\begin{enumerate}[(1)]
 \item $A$ is $\Sigma_n \Leftrightarrow \Conj{A}$ is $\Pi_n$
 \item $A$ is $\Sigma_n$ => $\forall m > n$, $A$ is $\Sigma_m$ and $\Pi_m$, 
	introducing ``vacuous'' quantifiers.
 \item $A,B$ is $\Sigma_n$( or $\Pi_n$)	=> $A \cup B$, $A \cap B$ is $\Sigma_n$
 \item \emph{quantifier contraction}:
	$R$ is $\Sigma_n	=> A = \{x : \exists y R(x,y) \}$ is $\Sigma_n$
 \item $B \le_m A$, $A$ is $\Sigma_n$ => $B$ is $\Sigma_n$.
   \begin{proof}
	If $A = \{ x : \exists y_1 \forall \ldots R(x, \vec{y}) \}$
	and $B \le_m A$ via f,
	then $B = \{ x : \exists y_1 \forall \ldots R(f(x), \vec{y}) \}$
      \end{proof}

 \item If R is $\Sigma_n$ (or $\Pi_n$), then
	\begin{align*}
          A = \{ <x, y> : (\forall z < y) R(x,y,z) \} \\
          B = \{ <x, y> : (\exists z < y) R(x,y,z) \} \\
	\end{align*}
	are $\Sigma_n$ (or $\Pi_n$)
		\proof (Exercise) Lookat pf of $A$ c.e. iff $A$ $\Sigma_1$-def'able over $\Nat$.

\end{enumerate}



\end{document}

