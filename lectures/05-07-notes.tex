\documentclass[12pt]{article}

\usepackage[centertags]{amsmath}
\usepackage{amsthm}
\usepackage{amssymb}
\usepackage{amsfonts}
\usepackage{amscd,amsbsy}
\usepackage{txfonts}
\usepackage[T1]{fontenc}
\usepackage{fullpage}
\usepackage{ulsy}

\usepackage{eucal}
\usepackage{enumerate}
\usepackage{pdfsync}
\usepackage{undertilde}

\newcommand{\Nat}{\ensuremath{\mathbb{N}}}
\newcommand{\Integer}{\ensuremath{\mathbb{Z}}}
\newcommand{\Rat}{\ensuremath{\mathbb{Q}}}
\newcommand{\Real}{\ensuremath{\mathbb{R}}}
\newcommand{\Comp}{\ensuremath{\mathbb{C}}}
\newcommand{\Conj}[1]{\ensuremath{\overline{#1}}}

\newcommand{\Lang}{\ensuremath{\mathcal{L}}}

\newcommand{\LA}{\ensuremath{\mathcal{L}_A}}
\newcommand{\M}{\ensuremath{\mathcal{M}}}
\newcommand{\TA}{\ensuremath{\operatorname{Th}(\Nat)}}
\newcommand{\PA}{\ensuremath{\mathsf{PA}}}

\newcommand{\Term}[1]{\ensuremath{\underline{#1}}}

\newcommand{\Pow}[1]{\ensuremath{\mathcal{P}(#1)}}

\newcommand{\forces}{\Vdash}
\newcommand{\proves}{\vdash}
\newcommand{\gn}[1]{\ulcorner #1 \urcorner}
\newcommand{\defn}{\textbf{Definition}: }
\newcommand{\fixme}{\\ \textbf{FIXME}: Incomplete notes!}


\author{Ben Lickly}
\date{May 7, 2009}
\title{Math 225B Notes}
\begin{document}
\maketitle

Idea: Pick a countable family of measure 1. sets.

PROP: The intersection of countable many meas. 1 sets has meas. 1.
\begin{proof}
  Show that countable union of nullsets is null.
  Let $U_1,..$ be meas. 0 sets.

  For each $n$, choose $\left( W^{(n)}_k \right)_{K\in \Nat}$ s.t.
\begin{align}
  U_n \subseteq \bigcup_{\sigma\in W^{(n)}_k} [\sigma]
  \text{ and }
  \sum_{\sigma\in W^{(n)}_k} 2^{-|\sigma|} < 2^{-k}
  \label{eq:plus}
\end{align}
Let $V_n = \bigcup_{k=1}^\infty W^{(k)}_{n+k}$.

Since $U_n \subseteq \bigcup_{W^{(n)}_l} [\sigma]$ f.all $l$,
we have:
\begin{align*}
  \bigcup_{n=1}^\infty U_n \subseteq
  \bigcup_{k=1}^\infty \bigcup_{\sigma\in W^{(k)}_{m+k}}[\sigma] \\
  \text{and }
  \sum_{\sigma \in V_m} 2^{-|\sigma|} = \sum_{k=1}^\infty \sum_{\sigma\in W^{(k)}_{m+k}} 2^{-|\sigma|}
  \\
  \le
  \sum_{k=1}^\infty 2^{-m-k} = 2^{-m}
\end{align*}
\end{proof}

\defn A \emph{Martin-L\"of test} is a c.e. set $W \subseteq \Nat \times 2^{<\Nat}$ s.t. f.all $n$,
\[
\sum_{(n,\sigma) \in W} < 2^{-n}
\]

Think: $W$ induces a seqn. of uniformly c.e. sets:
\[
W_n = \left\{ \sigma : (n,\sigma) \in W \right\}
\]
A sequence $X \in 2^W$ is \emph{convered by a test $W$} if f.all $n$,
$X \in \bigcup_{\sigma \in W_n} [\sigma]$.

\defn $X$ is \emph{Martin-L\"of racdom} if it is not covered by any ML-test.

By previous proposition, ML-random seqn. exist.
(there are only countabl many ML-tests.)

PROP:
No computable sequence is random.
Gicen $X \in 2^\Nat$ computable.
\begin{proof}
  Take the test $W_n = \left\{ x\lceil n+1 \right\}$
  $W = \left\{ (n, x\lceil n+1) : n \in  \Nat \right\}$
\end{proof}

Exercise: Show that no c.e. set is random.

Universal test:  There exists a test $U$ such that
$X$ is ML-random iff it is not covered by $U$.
\begin{proof}[idea]
  First enumerate all ML-tests

  Then perform the ``union of null sets is null'' construction.
\end{proof}
COR: The set of all ML-random sequences
is an effective union of $\Pi^0_1$-classes.
i.e. there exists a comp. function
$f: \Nat \rightarrow \Nat$
s.t. f.all $n$, $f(n)$ is an index of a comput. tree
$T_{f(n)}$ and $X$ ML-random iff $X \in \bigcup_{n=0}^\infty [T_{f(n)}]$
%                                        ^ $\Sigma^0_2$-class
\begin{proof}
  Let $U^{(n)} = \left\{ \tau : \exists \sigma \left( (n,\sigma) \in U \right) \wedge \tau \supset \sigma \right\}$
  where $u$ is the universal ML-test.

  $\left( U^{(n)} \right)_{n \in \Nat}$ is a seqn. of uniformly c.e. sets.
  The complement of each $U^{(n)}$ induces a $\Pi^0_1$-class $\mathcal{P}^{(n)}$we can unoformly conert an index for $U^(n)$ into an index for a comp. tree repres of $P^{(n)}$.
\end{proof}

COR: Low Basis Thm $\Rightarrow$ There exist low ML-random sequences.


However:

THM: (Kucera, Gacs)
For every $\underline{Y} \in 2^\Nat$ there ex. ML-rand. $X$ s.t. $Y \le_T X$.

We saw (HW): If $\underline{Y} >_T \sigma,\ \lambda\left\{ X : Y \le_T X \right\} = 0$
%
THM says that this does have effective measure 0

\section{Kolomogorov Complexity}
Let $M$ be a Turing machine.
Def. the \emph{M-complexity} of a string $\sigma \in 2^{<\Nat}$ as
\[                       % p \in 2^{<\Nat}
C_M(\sigma) = min\left\{ |p| : M(p) \downarrow = \sigma \right\}
\]
If there is no-such $p$, let $C_M(\sigma) = \infty$


Equivariance THm: There ex. machine $U$ s.t. f.all other machines, there
ex. constant $\gamma_M$ s.t.
\[
\forall \sigma C_U(\sigma) \le C_M(\sigma) + \gamma_M
\]
\begin{proof}
  Use universal TM:
  $\gamma_M$ is given by GN of M
\end{proof}
Fix $U$
Define $C(\sigma) = C_U(\sigma)$, the \emph{Kolomogorov Complexity} of $\sigma$.

Call $\sigma$ \emph{random} if $C(\sigma) > \sigma$.

Counting argument: For all $n$, there ex. $\sigma = n$
and $\sigma$ is random.

\begin{align*}
  2^n \text{ strings of length } n \\
  2^n - 1 \text{ strings of length } < n
\end{align*}

\defn $X \in 2^\Nat$ is incompressible if
f.all $n$, $C(X \lceil n) \ge n$
%
Such $X$ does not exist.
Reason: length of input provides extra information.


\defn TM $M$ is called \emph{prefix-free} if whenever
$M(\sigma)\downarrow$ then $M(\tau) \uparrow$ f.all $\tau \supset \sigma$

Similar to above, there exists a universal prefix-free TM $\widetilde{U}$<++>
\begin{align*}
  \widetilde{U}
  C_{\widetilde{U}}(\sigma) \le_M(\sigma) + \gamma_M
\end{align*} f.all prefix-free machines $M$.

Define the \emph{prefix-free Kolomogorov complexity} of $\sigma$
as $K(\sigma) = C_{\widetilde{U}}(\sigma)$.

THM:
$X$ is ML-random iff there ex. a $c$ s.t.
\[
\forall n, K(X\lceil n) \ge n - c
\]
We have
\[
K(\sigma) \le C(\sigma) + 2\log C(\sigma) + O(1)
\]
\end{document}
