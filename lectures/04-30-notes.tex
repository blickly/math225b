\documentclass[12pt]{article}

\usepackage[centertags]{amsmath}
\usepackage{amsthm}
\usepackage{amssymb}
\usepackage{amsfonts}
\usepackage{amscd,amsbsy}
\usepackage{txfonts}
\usepackage[T1]{fontenc}
\usepackage{fullpage}
\usepackage{ulsy}

\usepackage{eucal}
\usepackage{enumerate}
\usepackage{pdfsync}
\usepackage{undertilde}

\newcommand{\Nat}{\ensuremath{\mathbb{N}}}
\newcommand{\Integer}{\ensuremath{\mathbb{Z}}}
\newcommand{\Rat}{\ensuremath{\mathbb{Q}}}
\newcommand{\Real}{\ensuremath{\mathbb{R}}}
\newcommand{\Comp}{\ensuremath{\mathbb{C}}}
\newcommand{\Conj}[1]{\ensuremath{\overline{#1}}}

\newcommand{\Lang}{\ensuremath{\mathcal{L}}}

\newcommand{\LA}{\ensuremath{\mathcal{L}_A}}
\newcommand{\M}{\ensuremath{\mathcal{M}}}
\newcommand{\TA}{\ensuremath{\operatorname{Th}(\Nat)}}
\newcommand{\PA}{\ensuremath{\mathsf{PA}}}

\newcommand{\Term}[1]{\ensuremath{\underline{#1}}}

\newcommand{\Pow}[1]{\ensuremath{\mathcal{P}(#1)}}

\newcommand{\forces}{\Vdash}
\newcommand{\proves}{\vdash}
\newcommand{\gn}[1]{\ulcorner #1 \urcorner}
\newcommand{\defn}{\textbf{Definition}: }
\newcommand{\fixme}{\\ \textbf{FIXME}: Incomplete notes!}

\author{Ben Lickly}
\date{April 30, 2009}
\title{Math 225B Notes}
\begin{document}
\maketitle

$\PA \not\proves (*)$

$\mathcal{M}$ nonstandard model of $\PA + (*)$
\fixme

Consider little variant
\[
(\tau = )
\forall (p \text{ codes set of formulas and GN's of these formulas $< c$})
\]
In $\mathcal{M}$, this can be a non-standard element hence we can code
infinite sets of formulas.
% ^ in the ``real world''

Such ``infinite'' codes must exist b/c of ``overspill''.

It also holds that $\mathcal{M} \models \tau$

Therefore, by choice of $d$ (and the fact that $\mathcal{M} \models (*)$,
there ex. $I \subseteq [c,d)$, $|I| \ge c$
s.t. f.all $\Sigma_1$-formulas with GN $< c$
and f.all $i_0 < i_1 < \dots < i_n$ and $i_0 < j_1 < \dots < j_n$.
\[
\mathcal{M} \models \forall y < x_{i_0} \left( \varphi(y,x_{i_1},\dots,x_{i_n})
\leftrightarrow \varphi(y,x_{j_1},\dots,x_{j_n})\right)
\]
i.e. $\mathcal{M}$ thinks $I$ is a set of indiscernibles for all
$\Sigma_1$-formulas with GN $< c$.
By absoluteness of $\Delta_1$ properties, these include all standard $\Delta_0$-formulas.

Let $\mathcal{Y} = \left\{ x_i : i < w \right\}$ be an initial segment of $I$
and let $\mathcal{N}$ be the initial segment of $\mathcal{M}$
with universe $N = \left\{ y\in M : y < x_i \text{ for some } x_i \in \mathcal{Y} \right\}$
\begin{itemize}
  \item By Thm ``Models from indiscernibles,'' $\mathcal{N} \models \PA$
  \item We have $c \in N$, but $d \not\in N$
    (this will yield that $\mathcal{N} \not\models (*)$)

    Argument: if (*) were true, then it would hold for some $d'<d$.
    But this would work also for $\mathcal{M}$
    $\Rightarrow\Leftarrow$ (choice of $d$ least such)
\end{itemize}


\end{document}
