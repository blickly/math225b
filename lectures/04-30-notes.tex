\documentclass[12pt]{article}

\usepackage[centertags]{amsmath}
\usepackage{amsthm}
\usepackage{amssymb}
\usepackage{amsfonts}
\usepackage{amscd,amsbsy}
\usepackage{txfonts}
\usepackage[T1]{fontenc}
\usepackage{fullpage}
\usepackage{ulsy}

\usepackage{eucal}
\usepackage{enumerate}
\usepackage{pdfsync}
\usepackage{undertilde}

\newcommand{\Nat}{\ensuremath{\mathbb{N}}}
\newcommand{\Integer}{\ensuremath{\mathbb{Z}}}
\newcommand{\Rat}{\ensuremath{\mathbb{Q}}}
\newcommand{\Real}{\ensuremath{\mathbb{R}}}
\newcommand{\Comp}{\ensuremath{\mathbb{C}}}
\newcommand{\Conj}[1]{\ensuremath{\overline{#1}}}

\newcommand{\Lang}{\ensuremath{\mathcal{L}}}

\newcommand{\LA}{\ensuremath{\mathcal{L}_A}}
\newcommand{\M}{\ensuremath{\mathcal{M}}}
\newcommand{\TA}{\ensuremath{\operatorname{Th}(\Nat)}}
\newcommand{\PA}{\ensuremath{\mathsf{PA}}}

\newcommand{\Term}[1]{\ensuremath{\underline{#1}}}

\newcommand{\Pow}[1]{\ensuremath{\mathcal{P}(#1)}}

\newcommand{\forces}{\Vdash}
\newcommand{\proves}{\vdash}
\newcommand{\gn}[1]{\ulcorner #1 \urcorner}
\newcommand{\defn}{\textbf{Definition}: }
\newcommand{\fixme}{\\ \textbf{FIXME}: Incomplete notes!}

\author{Ben Lickly}
\date{April 30, 2009}
\title{Math 225B Notes}
\begin{document}
\maketitle

$\PA \not\proves (*)$

$\mathcal{M}$ nonstandard model of $\PA + (*)$
\fixme

Consider little variant
\[
(\tau = )
\forall (p \text{ codes set of formulas and GN's of these formulas $< c$})
\]
In $\mathcal{M}$, this can be a non-standard element hence we can code
infinite sets of formulas.
% ^ in the ``real world''

Such ``infinite'' codes must exist b/c of ``overspill''.

It also holds that $\mathcal{M} \models \tau$

Therefore, by choice of $d$ (and the fact that $\mathcal{M} \models (*)$,
there ex. $I \subseteq [c,d)$, $|I| \ge c$
s.t. f.all $\Sigma_1$-formulas with GN $< c$
and f.all $i_0 < i_1 < \dots < i_n$ and $i_0 < j_1 < \dots < j_n$.
\[
\mathcal{M} \models \forall y < x_{i_0} \left( \varphi(y,x_{i_1},\dots,x_{i_n})
\leftrightarrow \varphi(y,x_{j_1},\dots,x_{j_n})\right)
\]
i.e. $\mathcal{M}$ thinks $I$ is a set of indiscernibles for all
$\Sigma_1$-formulas with GN $< c$.
By absoluteness of $\Delta_1$ properties, these include all standard $\Delta_0$-formulas.

Let $\mathcal{Y} = \left\{ x_i : i < w \right\}$ be an initial segment of $I$
and let $\mathcal{N}$ be the initial segment of $\mathcal{M}$
with universe $N = \left\{ y\in M : y < x_i \text{ for some } x_i \in \mathcal{Y} \right\}$
\begin{itemize}
  \item By Thm ``Models from indiscernibles,'' $\mathcal{N} \models \PA$
  \item We have $c \in N$, but $d \not\in N$
    (this will yield that $\mathcal{N} \not\models (*)$)

    Argument: if (*) were true, then it would hold for some $d'<d$.
    But this would work also for $\mathcal{M}$
    $\Rightarrow\Leftarrow$ (choice of $d$ least such)
\end{itemize}

\section{$\Pi^0_1$-classes (Effectively Closed Sets)}

Recall: Topology on $2^\Nat$:
Basis $[\sigma] = \left\{ X \in 2^\Nat : X \supset \sigma \right\}$

Basic properties:
\begin{itemize}
  \item $2^\Nat$ compact
  \item totally disconnected ($[\sigma]$ is clopen)
\end{itemize}

Similar: $\Nat^\Nat$, not compacr
$\left\{ [<n>] \right\}_{n \in \Nat}$ % string over \Nat of length 1.
does not have a finite subcover.

Open Sets: Unions of cylinders

Open representation of open set $U$ is given by a $W \subseteq 2^{<\Nat}$
s.t.
\[
\bigcup_{\sigma \in W} [\sigma] = U
\]
We can assume that $W$ is closed upwards:
If $\sigma \in W$ and $\sigma \subset \tau \Rightarrow \tau \in W$.

By using a standard computable bijection $2^{<\Nat} \leftrightarrow \Nat$
we can apply computability theoretic notions to $W$.

Closed Sets:
$\mathcal{C}$ complement of open set, $\mathcal{C} = 2^\Nat \setminus U$

If $W$ is an open representation of $U$,
then $F = 2^{<\Nat} \setminus W$ is called a \emph{closed representation}
of $\mathcal{C}$

Observation:
If $F = 2^{<\Nat} \setminus W$ is a closed represent then $F$ is a tree
(in the simple sense: $\sigma \in T \& T \subset \sigma \Rightarrow \tau \in T$)
($W$ is closed upwards)

Furthermore, $\mathcal{C} = [F]$.

\subsection{Effectively open and closed sets}

Let $U$ be open.  $U$ is \emph{effectively open} or
\emph{$\Sigma^0_1$} (lightface) if it has a $c.e.$ open representation.

$\mathcal{C} \subseteq 2^{<\Nat}$ is \emph{effectively closed} or
a \emph{$\Pi^0_1$-class} if it's complement is effectively open.

It follows from the definition that $\mathcal{C}$ is $\Pi^0_1$-class
iff there exists a co-c.e. set $F$ s.t. $[F] = \mathcal{C}$.

Both notinos can be reativised:
%
Given $z \in 2^\Nat$, say $\mathcal{U} \subseteq 2^\Nat$ is $\Sigma^0_1(z)$
if it has an open representation that is c.e. in $z$.
%
Similar for closed sets.

Observation: The open (\textbf{$\Sigma^0_1$} (closed (\textbf{$\Pi^0_1$})
sets of $2^\Nat$ are exactly the ones which are $\Sigma^0_1(z)$
($\Pi^0_1(z)$) for some $z \in 2^\Nat$

\newtheorem*{thm}{THM}
\begin{thm}
  $\mathcal{C}$ is $\Pi^0_1$ iff there exists a computable tree $T$
  s.t. $\mathcal{C} = [T]$
\end{thm}
\begin{proof}
  $(\Leftarrow)$ clear, since every computable set is co-r.e.

$(\Rightarrow)$ Assume $\mathcal{C} = [F]$, $F \subseteq 2^{<\Nat}$ co-c.e. tree

By Normal Form,
\[
F = \left\{ \sigma : \forall n\ R(n, \sigma) \right\}
\]

\defn
\[
T = \left\{ \sigma : \forall n<|\sigma|
\forall \tau \subseteq \sigma\ R(n,\tau) \right\}
\]
Then $T$ is computable.

Claim: $[T] = [F]$

CLear: $[T] \supseteq [F]$ since $T \supseteq F$

Assume $X \not\in [F]$.
Then there exists $n$ s.t.
\[
\forall l \ge n X \lceil l \not\in F
\]
Hence $\forall l \ge n$ there exists $m_l$ s.t.
$R(m_l, X \lceil l)$ does not hold.

Choose $l > m_n$.  THen $X \lceil l \not\in T$
\end{proof}

\subsection{Examples}
\begin{enumerate}
  \item Given a Turing functinoal $\Phi$, let
\[
\mathcal{C} = \left\{ z \in 2^\Nat : \Phi^z(n) \uparrow \right\}
\]
We have $\mathcal{C} = [T]$ where $T = \left\{ \sigma:\forall s \Phi^\sigma_s(n) \uparrow \right\}$
  \item Let $C,D \subset \Nat$ be disjoint.
    A \emph{separating set} for $C,D$ is a set $A \subseteq \Nat$
    s.t. $C \subseteq A$, $A \cap D = \emptyset$

If $C,D$ are disjoint c.e. sets, then the set
\[
\left\{ A \in 2^\Nat : A \text{ separates } C,D \right\}
\]
is a $\Pi^0_1$-class.
\begin{proof}
  $A = [T], C = W_e, D = W_i $ where
  $T = \left\{ \sigma : \forall n < |\sigma| \forall s
  \left( W_{e,s}(n) \downarrow \Rightarrow \sigma(n) = 1 \right) \&
  \left( W_{i,s}(n) \downarrow \Rightarrow \sigma(n) = 0 \right)
  \right\}$
\end{proof}

  \item Let $\Gamma$ be an effectively axiomatizable,
    consistent theory.  Then the set of consistent extensions of $\Gamma$
    is a $\Pi^0_1$-class
\begin{proof}
  Fix eff. enumeration of sentences: $\sigma_0, \sigma_1, \dots$

  Let $C = \left\{ i : \Gamma \proves \sigma_i \right\}$
  and $D = \left\{ j : \Gamma \proves \neg\sigma_j \right\}$
  These are disjoint c.e. sets.

  Refine $\tilde{T}$ by $\sigma \in \tilde{T} \Leftrightarrow \Sigma$ is in
  the tree of Exa. (2)
  \& $\forall i < |\sigma|$ ( Set of formulas def'd by $\sigma(i) = 1$ is consistent)
\end{proof}

 \item Complete extensions of $\Gamma$
\end{enumerate}

\subsection{Basis Theorems for $\Pi^0_1$-classes}
Let $A$ be a class of subsets of $\Nat$ (Think: property)

$A$ is called a \emph{basis for $\Pi^0_1$-classes} if every non-empty
$\Pi^0_1$-class has a member that is in $A$.

Negative result:
The computable sets do not form a basis for $\Pi^0_1$-classes
\begin{proof}
  By (3), the consistent extensions of $\PA$ form a $\Pi^0_1$-class.
  By First Incomp. Thm, no member is complete.

  \defn Two disjoint c.e. sets $C,D$ are called \emph{computable inseparable}
  if there does not exist a computable separating set.

  COR: There ex. comp. insep. c.e. sets $C,D$
\end{proof}


\section{Next Time}

\newtheorem*{krem}{Kremel Basis Thm}
\begin{krem}
  $\Delta^0_2$ is basis
\end{krem}
\newtheorem*{low}{Low Basis Thm}
\begin{low}
  \[
X : X' \le_T \emptyset'
  \]
\end{low}

Introduce $ML$-random sequences
\end{document}
