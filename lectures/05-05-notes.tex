\documentclass[12pt]{article}

\usepackage[centertags]{amsmath}
\usepackage{amsthm}
\usepackage{amssymb}
\usepackage{amsfonts}
\usepackage{amscd,amsbsy}
\usepackage{txfonts}
\usepackage[T1]{fontenc}
\usepackage{fullpage}
\usepackage{ulsy}

\usepackage{eucal}
\usepackage{enumerate}
\usepackage{pdfsync}
\usepackage{undertilde}

\newcommand{\Nat}{\ensuremath{\mathbb{N}}}
\newcommand{\Integer}{\ensuremath{\mathbb{Z}}}
\newcommand{\Rat}{\ensuremath{\mathbb{Q}}}
\newcommand{\Real}{\ensuremath{\mathbb{R}}}
\newcommand{\Comp}{\ensuremath{\mathbb{C}}}
\newcommand{\Conj}[1]{\ensuremath{\overline{#1}}}

\newcommand{\Lang}{\ensuremath{\mathcal{L}}}

\newcommand{\LA}{\ensuremath{\mathcal{L}_A}}
\newcommand{\M}{\ensuremath{\mathcal{M}}}
\newcommand{\TA}{\ensuremath{\operatorname{Th}(\Nat)}}
\newcommand{\PA}{\ensuremath{\mathsf{PA}}}

\newcommand{\Term}[1]{\ensuremath{\underline{#1}}}

\newcommand{\Pow}[1]{\ensuremath{\mathcal{P}(#1)}}

\newcommand{\forces}{\Vdash}
\newcommand{\proves}{\vdash}
\newcommand{\gn}[1]{\ulcorner #1 \urcorner}
\newcommand{\defn}{\textbf{Definition}: }
\newcommand{\fixme}{\\ \textbf{FIXME}: Incomplete notes!}


\author{Ben Lickly}
\date{May 5, 2009}
\title{Math 225B Notes}
\begin{document}
\maketitle
$\Pi^0_1$-classes
set of infinite paths through computable trees

\section{Basis Theorems}
Family of sets s.t. every non-empty $\Pi^0_1$-class has a member from that
family.

Comp sets do not form a basis.

\newtheorem*{krb}{Kreisel Basis Thm}
\begin{krb}
  The $\Delta^0_2$ sets form a basis for $\Pi^0_1$<++>-classes
\end{krb}
\begin{proof}
  Linear order on $2^\Nat$:
  If $X,\underline{Y} \in 2^\Nat, X\ne \underline{Y}$ then say
  $X < \underline{Y}$ if $X(N) < \underline{Y}(N)$
  where $N = min_n\left\{ X(n) \ne Y(n) \right\}<++>$<++>
  We define the \emph{leftmost path} of $[T]$ as the $<$-min element of $[T]$.

  Claim: This exists

  We define seqn. $X \in 2^\Nat$ as follows:
  % Or do Konig's Lemma but always pick leftmost possible extension
  Let $X \lceil n$ = the lexicograph. least $\sigma$
  extending $X \lceil n-1$, $|\sigma| = n$
  s.t. $T_\sigma = \left\{  \tau \in T : \sigma \subseteq T \right\}$
  is infinite.
  By def, $X$ is a leftmost path.

  Claim 2: $X$ is $\Delta^0_2$
  \begin{proof}
    By construction of $X$, we need to determine whether for given $\sigma$,
    the tree above $\sigma$ is infinite.
\[
T_\sigma \text{ infinite } \Leftrightarrow \forall n \ge |\sigma|
        (\exists \tau, |\tau| = n) (\tau \supseteq \sigma \& \tau \in T)
\]
    This is $\Pi^0_1$ so $\emptyset'$ can decide this.
  \end{proof}
\end{proof}

\subsection{Low Basis Theorm}

A set $A$ is \emph{low} if $A' \le_T \emptyset'$.
\newtheorem{lowbasis}{Low Basis Thm}
\begin{lowbasis}
  Every non-empty $\Pi^0_1$-class has a low member.
\end{lowbasis}
Cor: There are non-comp low sets.
\begin{proof}
  Let $\mathcal{P}$ be $\Pi^0_1$ and non-empty.
  Define a sequence of $\Pi^0_1$-classes:
  \[
  \mathcal{P} = \mathcal{P}_0 \supseteq \mathcal{P}_1 \dots
  \]
  $\mathcal{P}_i$ are non-empty.

  By compactness, $\widetilde{\mathcal{P}} = \bigcap_{i=0}^\infty \mathcal{P}_i$ is non-empty.

  Given $\mathcal{P}_n$, define $\mathcal{P}_{n+1}$

  Case 1: $\mathcal{P}_n \cap U_n \ne \emptyset$,
  where $U_n = \left\{ Z : \Phi^z_n(n) \uparrow \right\}$
  %      ^ is \Pi^0_1
  Then let $\mathcal{P}_{n+1} = \mathcal{P}_n \cap U_n$
  (Note: Intersection of two $\Pi^0_1$-classes is $\Pi^0_1$.)

  Case 2: o.w.
  Then $\Phi^z_n(n) \downarrow$ f.all $Z \in \mathcal{P}_n$.
  Let $\mathcal{P}_{n+1} = \mathcal{P}$

  Claim: For any $Y \in \widetilde{P}, Y' \le_T \emptyset'$
  \begin{proof}
\begin{itemize}
  \item
    First observe that fiven a comp. tree $T$,
    $\emptyset'$ can determine whether $[T] \ne \emptyset$
    (As in proof of Kreisel Basis Thm).
  \item
    The intersection of two comp. trees is a computable tree.
  \item Finally, we can use $\emptyset'$ to effectively determine
    an index for each tree occurring in the construction.
\end{itemize}
  \end{proof}
  $\emptyset'$ determines whether case 1 or case 2 applies.

  Case 1 $\Rightarrow n \not\in \underline{Y}'$

  Case 2 $\Rightarrow n \in \underline{Y}'$
\end{proof}

\section{Martin-L\"of Randomness}

Recall: Lebesgue measure on $2^\Nat$:
$A \subseteq 2^\Nat$ is \emph{$\lambda$-null} or has \emph{$\lambda$-measure 0} if
for any $\varepsilon > 0$ there ex. $W \subseteq 2^{<\Nat}$ s.t.
\begin{align*}
  A \subseteq \bigcup_{\sigma \in W} [\sigma]
\quad \text{and} \quad
  \sum_{\sigma \in W} 2^{-|\sigma|} < \varepsilon
\end{align*}

Reformulation: $A$ has measure 0 iff there exists a sequence
$\left( W_n \right)_{n \in \Nat}$ s.t. f.all $n$:
\[
  A \subseteq \bigcup_{\sigma \in W} [\sigma]
\quad \text{and} \quad
\sum_{\sigma \in W} 2^{-|\sigma|} < \varepsilon
\]

Goal: Define an \emph{individual random} sequence.
e.g. the outlaw of an infinite sequence of coin tosses.

Probability Theory:
The complement of the set
\[
\left\{ X : lim_{n \rightarrow \infty} \frac{|\left\{ i < n : X(i) = 1\right\}|}{n} = 1/2 \right\}
\]
has measure 0. (Law of large numbers)

Random sequences should satisfy all properties of measure 1.

Problem: Given $X \in 2^\Nat$, $\left\{ X \right\}$ is null.
Hence $2^\Nat \setminus \left\{ X \right\}$ has measure 1.
Therefore, there ex. a set of measure 1 to which $X$ does not belong.
\end{document}
