\documentclass[12pt]{article}

\usepackage[centertags]{amsmath}
\usepackage{amsthm}
\usepackage{amssymb}
\usepackage{amsfonts}
\usepackage{amscd,amsbsy}
\usepackage{txfonts}
\usepackage[T1]{fontenc} 
\usepackage{fullpage}
\usepackage{ulsy}

\usepackage{eucal} 
\usepackage{enumerate}
\usepackage{pdfsync}
\usepackage{undertilde}

\newcommand{\Nat}{\ensuremath{\mathbb{N}}}
\newcommand{\Integer}{\ensuremath{\mathbb{Z}}}
\newcommand{\Rat}{\ensuremath{\mathbb{Q}}}
\newcommand{\Real}{\ensuremath{\mathbb{R}}}
\newcommand{\Comp}{\ensuremath{\mathbb{C}}}
\newcommand{\Conj}[1]{\ensuremath{\overline{#1}}}

\newcommand{\Lang}{\ensuremath{\mathcal{L}}}

\newcommand{\LA}{\ensuremath{\mathcal{L}_A}}
\newcommand{\M}{\ensuremath{\mathcal{M}}}
\newcommand{\TA}{\ensuremath{\operatorname{Th}(\Nat)}}
\newcommand{\PA}{\ensuremath{\mathsf{PA}}}

\newcommand{\Term}[1]{\ensuremath{\underline{#1}}}

\newcommand{\Pow}[1]{\ensuremath{\mathcal{P}(#1)}}

\newcommand{\forces}{\Vdash}
\newcommand{\proves}{\vdash}
\newcommand{\gn}[1]{\ulcorner #1 \urcorner}
\newcommand{\defn}{\textbf{Definition}: }
\newcommand{\fixme}{\\ \textbf{FIXME}: Incomplete notes!}

\author{Ben Lickly}
\date{April 16, 2009}
\title{Math 225B Notes}
\begin{document}
\maketitle
.
\fixme

\section{Paris-Harrington Principle}
Variation of the Finite R.T.

For $n, m, r \in \Nat$, there ex. $l \in \Nat$ s.t.

If $[l]^n = C_1 \cup \dots \cup C_r$
there ehere sixists $\underline{Y} \subseteq \left\{ 0, \dots, l-1 \right\}$ s.t.
\begin{itemize}
  \item $\underline{Y}$ is homogeneous
  \item $|\underline{Y}| \ge m$
  \item $|\underline{Y}| \ge$ min $\underline{Y}$ (new) 
    ($\underline{Y}$ is ``relatively large'')
\end{itemize}


\begin{proof}
  As in FRT.

  Ass. there is  no such $l$.

  Let $T_e = \left\{ f : [e]^n \rightarrow r : \text{ no homog. $\underline{Y}$ exist with $|\underline{Y}| \ge m$, min $\underline{Y}$} \right\}$
  \fixme
\end{proof}

\newtheorem*{parhar}{Paris Harrington Theorem}
\begin{parhar}
  The Paris-Harrington Principle is not provable in $\PA$.
\end{parhar}
\begin{proof}
  (Idea)
\begin{enumerate}
  \item Show that downward closure of a set of diagonal indiscernibles in a model of $\PA$, $\mathcal{M}$, yields an initial segment of $\mathcal{M}$ that is itself a model of $\PA$
  \item Show that the Paris-Harrington Principle, or rather a consequence, (*), proves the existence of arbitrarily large (but finite) sets of diag. indiscernibles f.any finite number of formulas.
  \item Take a non-standard model of $\PA$ $\mathcal{M}$ such  that 
    $\mathcal{M} \models (*)$ and a nonst. element $c \in M$
  \item Use (*) to prove (in $\mathcal{M}$) to obtain a set $I$, $|I| \ge c$,
    of diaf. indisc. in $\mathcal{M}$ f.all $\Delta_0$-formulas with GN $< c$
    ($\mathcal{M}$ thinks there sre only finitely many).
  \item In part, $I$ is a set of diag. indisc. for all standard $\Delta_0$-formulas.
  \item $I$ induces a model of $\PA$, $\mathcal{N}$
  \item If $\PA \models (*)$ then (*) would have to hold.
  \item This will give a contradiction. 
\end{enumerate}
\end{proof}

\subsection{Terminology}
\subsubsection{Diagonal indiscernibles}
$\Gamma$ set of $\mathcal{L}_A$-formulas.
$\mathcal{M} \models \PA$

$I \subseteq \mathcal{M}$ sequence of diagonal indiscernibles for $\Gamma$
if whenever
\begin{itemize}
  \item \[\varphi(u_1, \dots, u_m, v_1, \dots, v_n) \in \Gamma\]
  \item \[z, x_1, \dots, x_n, y_1, \dots, y_n \in I \text{ with }
    z < x_1 < \dots < x_n \text{ and } z < y_1 < \dots < y_n\]
  \item and \[a_1,\dots,a_m < z\] 
\end{itemize}
then $\mathcal{M} \models \varphi[\bar{a}, x_1, \dots, x_n] \Leftrightarrow
\mathcal{M} \models \varphi[\bar{a}, y_1, \dots, y_n]$


We can use the Paris-Harrington Principle to obtain, \dots (as in 2.)

No hope for sets of inisc. that work for all fmlas.

\defn
Let $X \subset \Nat$
\begin{itemize}
  \item $f : [X]^n \rightarrow$ is \emph{regressive if}
    \fixme
  \item .\fixme
\end{itemize}

\textbf{
Combinatorial Principle (*)
}

For all $c,m,n,k$ there is a $d$ s.t. if 
$f_1, \dots, f_k : [d]^n \rightarrow d$ are regressive,
then there is $\underline{Y} \subset [c,d)$ s.t.
\[
|Y| \ge m
\]
and $\underline{Y}$ is min-homog. for each $f_i$

Exercise:
We have Par-Har Principle $\Rightarrow (*)$


\textbf{PROP:}
For any $c, l, m, n$ and any 
$\mathcal{L}_A$-formula 
$\varphi_1$
$\varphi_e$

there is a set $I$ fo diag indisc.
\fixme

\begin{proof}
  Wlog: $M > 2n$

  By FRT, ex. $W$ s.t. $W \rightarrow \left( m+n \right)^{2n+1}_{l+1}$

  $(*) \Rightarrow \exists s \ge l$ s.. whenever $f_1,\dots,f_k$  $[\bar{s}]^{2n+1} \rightarrow s$
  are regressive, there is $\underline{Y} \subseteq [c,s)$ with
  $|\underline{Y}| \ge w$ and $\underline{Y}$ min-homog. for each $f_i$
\end{proof}

\defn
regressive 
$f_i: [s]^{2n+1} \rightarrow l$ for $i = 1,\dots, k$
and
$g : [s]^{2n+1} \rightarrow l+1$

Let $X = \left\{ x_0 < x_1 < \dots < x_{2n} \right\}$   $x_{2n} < l$

If $\Nat \models \models \varphi_i[\vec{a}, x_1, \dots, x_n] \Leftrightarrow
\Nat \models \varphi_i[\vec{a}, x_{n+1}, \dots, y_{2n}]$
f.all $i \le l$ and  $a_1, \dots, a_m < x_0$
then let $f_i(X) = 0$ and $g(X) = 0$

Otherwise pick $i, a_1, \dots, a_m$ s.t.
\[
\varphi_i[\vec{a}, x_1, \dots, x_n] \not\Leftrightarrow_\Nat
\varphi_i[\vec{a}, x_{n+1}, \dots, y_{2n}]
\]
and put 
\begin{align*}
  g(X) = i \\
  f_i(X) = a_i
\end{align*}


Each $f_i$ regressive
$\underline{Y}$ min-homog. for each $f_i$
$|\underline{Y}| \ge w$

\textbf{Show} $g$ is $\equiv 0$ on $z$

pass to g-homog $Z \subset \underline{Y}$ $|Z| > 3n$

\[
\varphi_1[\vec{a}, x_1, \dots, x_n, x_{n+1}, \dots, x_{2n}] \not\Leftrightarrow_\Nat
\varphi[\vec{a}, x_1, \dots, x_{2n}, x_{2n+1}, \dots, x_{3n}]
\]

\end{document}
