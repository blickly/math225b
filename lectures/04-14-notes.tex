\documentclass[12pt]{article}

\usepackage[centertags]{amsmath}
\usepackage{amsthm}
\usepackage{amssymb}
\usepackage{amsfonts}
\usepackage{amscd,amsbsy}
\usepackage{txfonts}
\usepackage[T1]{fontenc} 
\usepackage{fullpage}
\usepackage{ulsy}

\usepackage{eucal} 
\usepackage{enumerate}
\usepackage{pdfsync}
\usepackage{undertilde}

\newcommand{\Nat}{\ensuremath{\mathbb{N}}}
\newcommand{\Integer}{\ensuremath{\mathbb{Z}}}
\newcommand{\Rat}{\ensuremath{\mathbb{Q}}}
\newcommand{\Real}{\ensuremath{\mathbb{R}}}
\newcommand{\Comp}{\ensuremath{\mathbb{C}}}
\newcommand{\Conj}[1]{\ensuremath{\overline{#1}}}

\newcommand{\Lang}{\ensuremath{\mathcal{L}}}

\newcommand{\LA}{\ensuremath{\mathcal{L}_A}}
\newcommand{\M}{\ensuremath{\mathcal{M}}}
\newcommand{\TA}{\ensuremath{\operatorname{Th}(\Nat)}}
\newcommand{\PA}{\ensuremath{\mathsf{PA}}}

\newcommand{\Term}[1]{\ensuremath{\underline{#1}}}

\newcommand{\Pow}[1]{\ensuremath{\mathcal{P}(#1)}}

\newcommand{\forces}{\Vdash}
\newcommand{\proves}{\vdash}
\newcommand{\gn}[1]{\ulcorner #1 \urcorner}
\newcommand{\defn}{\textbf{Definition}: }
\newcommand{\fixme}{\\ \textbf{FIXME}: Incomplete notes!}

\author{Ben Lickly}
\date{April 14, 2009}
\title{Math 225B Notes}
\begin{document}
\maketitle

\section{Ramsey's Thm}
\subsection{Notation}
$A$ is a set

\subsection{Ramsey's Thm}
\newtheorem{ramsey}{Ramsey's Thm}
\begin{ramsey}
  If $X$ is countably infinite and $[X]^n = C_1 \cup \dots \cup C_r$
  then $X$ contains an infinite homogeneous subset.
\end{ramsey}
\begin{proof}
  (By induction on $n$).

  $n=1$: Pigeonhole Principle

  $n+1$: Let $x_0 = min X$, put $\widetilde{X_0} = X \setminus \left\{ x_0 \right\}$
  Induce r.coloring of $[\widetilde{x_0}]^n$
  \[
  \left\{ y_1, \dots, y_n \right\} \in C^0_i \Leftrightarrow \left\{ x_0, y_1, \dots, y_n \right\} \in C_i
  \]
  By I.H., there ex. $X_1 \subseteq \widetilde{X_0}$ infinite homogeneous subset.

  Let $x_1 = min X_1$   $\widetilde{X_1} = X_1 \ \left\{ x_1 \right\}$
  induce r.coloring based on $x_1$

  $\rightarrow$ homog. subset $X_2 \subseteq \widetilde{X_1}$
  
  $\rightarrow^\text{inductively}$ infinite sequence $x_0 < x_1 < x_2 < \dots$
  sets $X_1 \supseteq X_2 \supseteq X_3 \supseteq \dots$

  By Pig. Pr. there exists $I$, $` \le I \le r$, s.t.
  $\left[ X_k \right]^n \subseteq C^{k-1}_i$ for $\infty$-many $k \ge 1$
  \fixme

  Claim $\bar{X}$ is homogeneous for the coloring.

  Let $y_1 < y_2 < \dots < y_{n+1} \in \bar{X}$.

  Then $y_1 = x_e$ for some $l \in \Nat$
  This implies $y_2 , \dots, y_{n+1} \in X_{e+1}$
  By choice of $\bar{X}$, $\left[ X_{e+1} \right]^n \subseteq C^e_i$

  By def of coloring $C^e_1 \cup \dots \cup C^e_r$, 
  \[
  \left\{ x_e, y_2, \dots, y_{n+1} \right\} \in C_i
  \]
\end{proof}

\section{Notation}

$[A]^n$ = $n$-dim subsets of $A$

$\kappa, \nu$ cardinals:
$\kappa \rightarrow (\nu)^n_r$ means
for every r-coloring of $[\kappa]^n$ there exists a homogeneous subset of size $\nu$

Ramsey's Thm:
\[
\aleph_0 \rightarrow (\aleph_0)^n_r \text{ for every $n, r \in \Nat$}
\]
One case show:
$2^{\aleph_0} \not \rightarrow (3)^2_{\aleph_0}$

\newtheorem*{finiteram}{Finite Ramsey Thm}
\begin{finiteram}
  For all $m, n, r \in \Nat$ there ex. $l \in \Nat$ s.t. $l \rightarrow (m)^n_r$ 
\end{finiteram}
\begin{proof}
  Sup no such $l$ exists.
  
  For each $l$, let
  $T_e = \left\{ f: \left[ \left\{ 0, \dots, l-1J \right\} \right]^n \rightarrow \left\{ 0, \dots, r-1 \right\}\right\}$
  s.t. no homogeneous subset of size $m$ exists

  Then $T_e \ne \emptyset$ for all $l$.
  Each $T_e$ is finite

  If $f \in T_{e+1}$, then there exists unique $g \subset f$ s.t. $g \in T_e$

  (Every coloring of $l+1$ induces coloring of $e$, and a homog. subset of $l$ is also homog. subset of $l+1$)

  Hence we can order $T = \bigcup_{e} T_e$ by inclusion and get a \emph{finite branching} infinite tree.


\subsubsection*{Aside}
  \defn A tree $T$ is \emph{finite branching} if for all $\sigma \in T$
  the set of immediate successors of $\sigma$ is finite.

\newtheorem*{konig}{K\"onig's Lemma (Compactness Principle for trees)}
\begin{konig}
  If $T$ is an infinite, finite branching tree, then
  \[
      [T] \ne \emptyset
  \] % ^ set of all infinte paths through $T$
\end{konig}
\begin{proof}
  Given $\sigma \in T$, let $[\sigma]^* = \left\{ \tau \in T : \tau \supseteq \sigma \right\}$

  Construct inf. path $X \in [T]$ inductively

  Let $X \lceil 0 = \emptyset$

  Given $X \lceil n$ s.t. $[ X \lceil n]^*$ is infinite,
  let $\left\{ \tau_1, \dots, \tau_m \right\}$ be the immediate successors of $X \lceil n$
  Since $[X \lceil n]^* = [\tau_1]^* \cup \dots \cup [\tau_m]^*$.

  By Pig.Pr., some $[\tau_i]^*$ is infinite

  Let $X \lceil n+1 = \tau_i$
\end{proof}

Pf of Fin R.T (cont.)
Let $F$ be an infinite path through $T$

Then $F$ represents $r$-coloring of $[\Nat]^n$

By Infinite Ramsey Thm, there ex. $H \subseteq \Nat$ homogeneous for $F$

Let $h_1 < h_2 < \dots < h_m$ be the first $m$ elements of $H$.

For $l > h_m$, $\left\{ h_1, \dots, h_m \right\}$ is homogeneous for $f_e \in T_e= F\lceil e$
\end{proof}

\subsection{Indiscernibles}

$\mathcal{L}$ language, $\mathcal{M}$ $\mathcal{L}$-structure

\defn $I$ infinite set, $A = \left\{ a_i : i \in I \right\} \subseteq \mathcal{M}$
($a_i$ disjoint)

$A$ is \emph{indiscernible} if f.all $i_1,\dots,i_m$ , $j_1, \dots, j_m \in I$
($i$s distinct from each other and $j$s distinct from each other)
then 
\[
\mathcal{M} \models \varphi\left[ a_{i,1}, \dots a_{im}\right] 
\Leftrightarrow \mathcal{M} \models \varphi\left[ a_{j,1}, \dots, a_{j,m} \right]
\]
f.all $\mathcal{L}$-formula $\varphi$

EXA: $F$ algebraically closed field infinite transcendence degree.

Think $\Comp$ over $\Rat$
infinite set $S \subseteq \Comp$ s.t.

$\forall a_1, \dots, a_n \in S$ f.all $p \in \Rat[x_1, \dots, x_n]$
\[
p(a_1, \dots, a_n) \ne 0
\]

By ZL, there ex. max. alg. independent set $A$

For $i_1, \dots, i_m$, $j_1, \dots, j_m$ there exists an automorphism
\[
\Rat(A) \rightarrow \Rat(A)
\]
s.t. 
$\sigma(a_{i,k}) = a_{j,k}$ and 
$\sigma = id$ on rest of $A$

One can show this extends to automorph. of $F$

Then $D \models \varphi[a_{i,1}, \dots, a_{i,m}] \Leftrightarrow F \models \varphi[\sigma(a_{i,1}), \dots, \sigma(a_{i,m})]$

\subsubsection*{Problem}
If $\mathcal{M}$ is ordered, it is hard to find an infinite set of indicernibles.

EXA: $(\Rat, <)$ For all $p,q \in \Rat$ $p \ne q$ either $p < q$ or $q < p$
$\rightarrow$ no set of indiscernibles of size 2


\defn
$(I, n)$ linear ordered set, $(a_i : i \in I)$ seqn. of distinct elements in $M$.

$(a_i : i \in I)$ is a seqn. of \emph{order indiscernibles}
if whenever $i_1 < i_2 < \dots < i_m$ and $j_1 < j_2 < \dots < j_m$
are two increasing seqn. in $I$, then
\[
\mathcal{M} \models \varphi[a_{i,1}, \dots, a_{i,m}]
\Leftrightarrow
\mathcal{M} \models \varphi[a_{j,1}, \dots, a_{j,m}]
\]
f.all $\mathcal{L}$-formulas $\varphi$

EXA: $Th(\Rat, <)$ has quantifier elimination
$\rightarrow$ For $\varphi$ exists quant. free $\psi$ s.t.
\[
\Rat \models \varphi \leftrightarrow \psi
\]
$\psi$ is essentially of the form
\[
x_{i,1} <\not=
x_{i,2} <\not=
\dots <\not=
x_{i,m}
\]

Let $I = \Rat$ 
$p_1 < \dots < p_m$,
$q_1 < \dots < q_m \in \Rat$,

Then
\[
\Rat \models \psi[p_1, \dots, p_m] \Leftrightarrow
\Rat \models \psi[q_1, \dots, q_m]
\]
$\rightarrow$ $\Rat$ is a set of indiscernibles for $\Rat$

\end{document}
