\documentclass[12pt]{article}

\usepackage[centertags]{amsmath}
\usepackage{amsthm}
\usepackage{amssymb}
\usepackage{amsfonts}
\usepackage{amscd,amsbsy}
\usepackage{txfonts}
\usepackage[T1]{fontenc} 
\usepackage{fullpage}
\usepackage{ulsy}

\usepackage{eucal} 
\usepackage{enumerate}
\usepackage{pdfsync}
\usepackage{undertilde}

\newcommand{\Nat}{\ensuremath{\mathbb{N}}}
\newcommand{\Integer}{\ensuremath{\mathbb{Z}}}
\newcommand{\Rat}{\ensuremath{\mathbb{Q}}}
\newcommand{\Real}{\ensuremath{\mathbb{R}}}
\newcommand{\Comp}{\ensuremath{\mathbb{C}}}
\newcommand{\Conj}[1]{\ensuremath{\overline{#1}}}

\newcommand{\Lang}{\ensuremath{\mathcal{L}}}

\newcommand{\LA}{\ensuremath{\mathcal{L}_A}}
\newcommand{\M}{\ensuremath{\mathcal{M}}}
\newcommand{\TA}{\ensuremath{\operatorname{Th}(\Nat)}}
\newcommand{\PA}{\ensuremath{\mathsf{PA}}}

\newcommand{\Term}[1]{\ensuremath{\underline{#1}}}

\newcommand{\Pow}[1]{\ensuremath{\mathcal{P}(#1)}}

\newcommand{\forces}{\Vdash}
\newcommand{\defn}{\textbf{Definition}: }
\newcommand{\fixme}{\\ \textbf{FIXME}: Incomplete notes!}

\author{Ben Lickly}
\date{March 19, 2009}
\title{Math 225 Notes}
\begin{document}
\maketitle

perfect trees: $T: 2^{<\Nat} \rightarrow 2^{<\Nat}$
\begin{align*}
  &R_e:& \chi_A \ne \varphi_e \\
  &Q_e:& \text{If } \Phi^A_e \text{ is total then either }
        \Phi^A_e \text{ is computable or } A \le_T \Phi^A_e
\end{align*}

\defn $(\sigma,\tau)$ is an \emph{e-splitting} of $\varrho$ if
  \[
  \varrho \subset \sigma,\tau
  \]
  and for some $x$, 
  $\Phi^\sigma_e(x)\downarrow \ne \Phi^\tau_e(x)\downarrow$

  A tree $T$ is \emph{e-splitting} if f.all $\varrho$
  \[
       \left(
        T(\sigma\frown 0), T(\sigma\frown 1)
        \right) \text{ is an e-splitting of $T(\sigma)$}
  \]
  % FIXME: Question: What would a not e-splitting tree look like?

  
\textbf{Crucial fact}
Sup $T \in \mathcal{T}_c$.
Let $A \in [T]$ s.t. $\Phi^A_e$ is  total.

Then
\begin{enumerate}[(1)]
  \item If there is no e-splitting on $T$,
    then $\Phi^A_e$ is computable.
  \item If $T$ is e-splitting, then $A \le_T \Phi^A_e$. 
\end{enumerate}

\begin{proof}
  (1) Given $x$, we have to compute $\Phi^A_e(x)$.
  
  Find $\sigma$ s.t. $\Phi^\varrho_e(x)\downarrow$.
  Such $\sigma$ must exist, since $\Phi^A_e$ is total
  and $A$ is an inf. path through $T$.

  There ex. $\tau$ on $T$, $\tau\subset A$ st. 
  $\Phi^\tau_e(x) \downarrow = \Phi^A_e(x) \downarrow$. 

  Since there are no e-splittings on $T$, we have 
  $\Phi^\sigma_e(x) = \Phi^\tau_e(x) = \Phi^A_e(x)$.
\end{proof}

\begin{proof}
  (2) Given $y \in \Nat$, we have to compute $A(y)$ 
  using $\Phi^A_e$ as an oracle.

  If $y < |T(\emptyset)|$, then since $T(\emptyset) \subset A$
  we can use the computability of $T$ to compute $A(y)$.

  Now assume we have found $\sigma \in 2^{<\Nat}$ s.t. $|\sigma| = n$
  and $T(\sigma) \subset A$.
  ( and hence can compute $A(y)$ for $ y < |T(\sigma)|$)

  By assumption 
      $ \left(
        T(\sigma\frown 0), T(\sigma\frown 1)
        \right) 
      $
      is an e-splitting for $T(\sigma)$

      We can eff. find $x$ (by dovetailing) s.t. 
      \[
        \Phi_e^{T(\sigma\frown 0)}(x)\downarrow 
    \ne \Phi_e^{T(\sigma\frown 1)}(x)\downarrow
      \]


      $\Phi_e^A(x)$ is equal to \textbf{exactly} one of 
        $\Phi_e^{T(\sigma\frown 0)}(x)$ and
        $\Phi_e^{T(\sigma\frown 1)}(x)$.

     By use principle, if $T(\sigma \frown i) \subset A$ and 
     $\Phi_e^{T(\sigma\frown i)}(x)\downarrow$, 
     then $\Phi_e^{T(\sigma\frown i)}(x) = \Phi^A_e(x)$.

     By comparing both $\Phi_e^{T(\sigma\frown i)}(x)$ 
     with the oracle $\Phi^A_e$,
     find which cof. $T(\sigma\frown 0)$, $T(\sigma\frown 1)$
     is an initial segment of $A$.
\end{proof}

\newtheorem*{splitlem}{LEMMA}
\begin{splitlem}
  Let $T \in \mathcal{T}_c$ and assume
  \begin{align*}
    \forall \varrho \exists \sigma, \tau ( (T(\sigma), T(\tau)) \text{ is an e-splitting of $T(\varrho)$})
  \end{align*}
  \label{eq:star}
  Then there ex. a comp subtree $T^*$ (call it $T^{SP}_e$) of $T$ that 
  is e-splitting.
\end{splitlem}
\begin{proof}
  Exercise
\end{proof}

Now we can prove density for $Q_e$ requirements.

\newtheorem*{lem2}{LEMMA}
\begin{lem2}
  Given $T \in \mathcal{T}_c$ and $e \in \Nat$, 
  there ex. $T^* \in \mathcal{T}_c$ s.t.
  $T^*$ is a subtree of $T$ and $T^* \forces Q_e$
\end{lem2}
\begin{proof}
  \textbf{Case 1}: There ex. $\sigma$ s.t. there are no 
  e-splittings above $\sigma$, i.e. on $T_\sigma$.

  Let $T^* = T_\sigma$.

  \textbf{Case 2}: No such $\sigma$ exists.  Then \ref{eq:star} holds and
  we lwt $T^* = T^{SP}_e$

  In both cases $T^*$ is a comp. subtree of $T$,
  and the crucial fact assures that $T^* \forces Q_e$.
\end{proof}

\newtheorem*{mintdeg}{THM}
\begin{mintdeg}
  THere exist minimal T-degrees
\end{mintdeg}
\begin{proof}
  For $R \in \mathcal{R}$ ($\mathcal{R}$ is a reqn. set)

  Let 
  $C_R = \left\{T\in \mathcal{T}_c : T \forces R \right\}$
  and let 
  $\mathcal{C} = \left\{C_R : R \in \mathcal{R} \right\}$
                % showed dense in p.o.

  There exists a $\mathcal{C}$-generic set $G \in \mathcal{T}_c$.
  
  Fix $A \in A_G$.
  Then $A$ is inf. path in $A_G$
\[
G \text{ generic } \rightarrow G \cap C_R \ne \emptyset \text{ f.all } R \in \mathcal{R}`
\]
\fixme

$A$ is a path through every tree in $G$
in particular, through a tree in $G \cap C_R$

$\Rightarrow A$ satisfies $R$.
\end{proof}

\section{Theories and Decidability}
G\"odel numbers of formulas.

\newcommand{\gn}[1]{\ulcorner #1 \urcorner}

\[
\mathcal{L}_A = \left\{+, *, < , 0, 1\right\}
\]

Terms:
\begin{align*}
  \gn{0} &=& 1 \\
  \gn{1} &=& 5 \\
  \gn{x_i} &=& 5^2 * 7 ^i \\
  \gn{s+t} &=& 5^3 * 7^{\gn{s}} * 11^{\gn{t}} \\
  \gn{s*t} &=& 5^4 * 7^{\gn{s}} * 11^{\gn{t}} \\
\end{align*}

Formulas:
\begin{align*}
  \gn{s=t} &=& 3 * 7^{\gn{s}} * 11^{\gn{t}} \\
  \gn{s<t} &=& 3 * 5 * 7^{\gn{s}} * 11^{\gn{t}} \\
  \gn{\neg \varphi} &=& 3 * 5^2 * 7^{\gn{\varphi}} \\
  \gn{\varphi \rightarrow \psi} &=& 3 * 5^3 * 7^{\gn{\varphi}} * 11^{\gn{\psi}} \\
  \gn{\exists x_i \varphi} &=& 3 * 5^4 * 7^i * 11^{\gn{\phi}} \\
\end{align*}

Sequences of formulas
\[
 \langle \varphi_0, \dots, \varphi_n \rangle 
 = p_0^{\gn{\varphi_0} + 1} * \dots * p_n^{\gn{\varphi_n} + 1}
\]

Convenient distinction between GV's of terms formuals, sqn of formulas.

\textbf{Important}: mapping is 1-1 and ``effective''
in particular set of GN's is a compu subset of $\Nat$

inverse is also effective

Possible for other countable languages


\textbf{Theories}
$T$ set of $\mathcal{L}_A$-sentences deductively closed.  
i.e. if $T \models \sigma$ then $\sigma \in T$.


$T$ is \emph{consistent} if $T$ has a model.

$T$ is \emph{effectively axiomatizable} if there ex a comp set of sentences 
$\Delta$ s.t.
\[
T = \left\{\sigma : \Delta \models \sigma \right\}
\]

T is \emph{complete} if f.all senteces,
\[
\sigma \in T \text{ or } \neg \sigma \in T
\]

Fix a complete decidable proof system for first order logic.
i.e.
\newcommand{\proves}{\vdash}
\[
\Gamma \proves \varphi \text{ iff } \Gamma \models \varphi
\]

$\Gamma$-proofs are finite sen. of formulas $<\varphi_0, \dots, \varphi_n >$.

\[
\varphi_k \in \Gamma \cap \text{Axioms}
\]
or there ex. $i,j < k$, s.t.
\[
\varphi_i = \varphi_j \rightarrow \varphi_k
\]

Assume $\Delta$ effectively axiomatized $T$

proof predicate:
\begin{align*}
  proof_T(x) \Leftrightarrow& x \text{ is GN of seqn. of formulas } 
  \wedge (\forall k > length(x) ) \\
  &\left[ (x)_k \in \Delta \cup \text{Axioms} \vee (\exists i,j < k) \left( (x)_i = \gn{(x)_i \rightarrow (x)_k}\right) \right]
\end{align*}

\defn
(bew = provable)

\begin{align*}
  bew(x,a) \Leftrightarrow proof_T(x) \wedge a =  (x)_{length(x)-1} \\
  bwb(a) \Leftrightarrow \exists z bew(x,a)
\end{align*}

\newtheorem*{cethm}{THM}
\begin{cethm}
  An effect. exiomatizable theory is c.e
\end{cethm}
\begin{proof}
  Follows from comp. of $bew(.,.)$
  and the normal form for c.e. sets.
\end{proof}
\newtheorem*{decthm}{THM}
\begin{decthm}
Every eff. axiom complete theory is decidable (= computable)
\end{decthm}
\begin{proof}
  Clear if $T$ is inconsistent.

  Sup $T$ is consistent.
  By completeness, $bwb(\neg \varphi) \Leftrightarrow \neg bwb(\varphi)$.

  Let $A = \left\{\varphi : bwb(\varphi)\right\}$ ($A$ is c.e.)
  $\Conj{A} = \left\{\varphi : \neg bwb(\varphi)\right\} = \left\{\varphi : bwb(\neg \varphi) \right\}
  $ c.e.

  $\Rightarrow $ is comp. by Complementation Thm.
\end{proof}

\newtheorem*{notdecthm}{THM}
\begin{notdecthm}
  $Th(\Nat)$ is not decidable.
\end{notdecthm}
\begin{proof}
  Every c.e. is $\Sigma_1$-def'able over $\Nat$
  Hence ther ex. $\varphi_k(x)$ s.t.
  \[
  \Nat \models \varphi_k[e] \text{ iff } e \in k
  \]
\end{proof}

\newtheorem*{cor}{COR}
\begin{cor}
  $Th(\Nat)$ is not eff. axiomatizable
\end{cor}
\begin{proof}
  $Th(\Nat)$ complete
\end{proof}

$PA$ is incomplete.

\end{document}
