\documentclass[12pt]{article}

%\usepackage{a4wide}
\usepackage[centertags]{amsmath}
\usepackage{amsthm}
\usepackage{amssymb}
\usepackage{amsfonts}
\usepackage{amscd,amsbsy}
\usepackage{txfonts}
\usepackage[T1]{fontenc} 

% \usepackage{eulervm}
\usepackage{eucal} 
%\usepackage{textcomp}

%\usepackage{mathptmx} 
\usepackage{enumerate}


\usepackage{pdfsync}


% Deutsche Texte
\usepackage[applemac]{inputenc}

% Seitenlayout
\pagestyle{empty}

\setlength{\textheight}{25cm}
\setlength{\textwidth}{16cm}

\setlength{\hoffset}{-1,5cm}
\setlength{\voffset}{-2,5cm}

\newcommand{\alabel}[1]{\mbox{(#1)}\hfill}
\newenvironment{alist}
        {\begin{list}{}%
                {\renewcommand{\makelabel}{\alabel}%
                \setlength{\parsep}{0.2cm}\setlength{\listparindent}{0cm}
                \setlength{\itemsep}{0.2cm}}}
        {\end{list}}

		\newcounter{aufgabenr}
\newcommand{\Aufg}[1][\mbox{}]{\subsubsection*{Problem \arabic{aufgabenr}#1}\stepcounter{aufgabenr}}
		\stepcounter{aufgabenr}


% Umgebung "Lösung"
\newenvironment{lsg}{\par\bigskip\noindent{\emph{Solution.\ }}\footnotesize}{\hfill $\blacksquare$}


% Zahlenmengen
\newcommand{\Nat}{\ensuremath{\mathbb{N}}}
\newcommand{\Integer}{\ensuremath{\mathbb{Z}}}
\newcommand{\Rat}{\ensuremath{\mathbb{Q}}}
\newcommand{\Real}{\ensuremath{\mathbb{R}}}
\newcommand{\Comp}[1]{\ensuremath{\overline{#1}}}
\newcommand{\Conj}[1]{\ensuremath{\overline{#1}}}

\newcommand{\Lang}{\ensuremath{\mathcal{L}}}

\newcommand{\LA}{\ensuremath{\mathcal{L}_A}}
\newcommand{\M}{\ensuremath{\mathcal{M}}}
\newcommand{\TA}{\ensuremath{\operatorname{Th}(\Nat)}}
\newcommand{\PA}{\ensuremath{\mathsf{PA}}}

\newcommand{\GN}[1]{\ensuremath{\ulcorner #1 \urcorner}}

\newcommand{\Term}[1]{\ensuremath{\underline{#1}}}

\newcommand{\Pow}[1]{\ensuremath{\mathcal{P}(#1)}}

\newcommand{\Ax}[1]{\ensuremath{\mathsf{#1}}}




\begin{document}

\begin{center}
{\LARGE Homework 9 for \textbf{MATH 225B}}

\vspace{0,5cm}

{\large Due: Thursday April 16}
\end{center}


\Aufg 
In class we showed that $\PA^{\bot} = \PA + \neg \Ax{CON}_{\PA}$ is consistent, but $\PA^{\bot}$ proves its own inconsistency.
Write two or three paragraphs explaining how one can resolve this apparent paradox.



\Aufg
Show that $\PA$ is interpretable in $\Ax{ZF}$.



\Aufg
Show that there is a strongly undecidable structure for a language  whose only non-logical symbol is a $4$-ary predicate symbol $R$.



\Aufg
Let $\Lang' \supset \Lang$ be an extension of a language $\Lang$ that adds only new constants. Suppose $\M'$ is an expansion of an $\Lang$-structure $\M$ to an $\Lang'$-structure. Show that if $\M'$ is strongly undecidable, then $\M$ is strongly undecidable.  



\Aufg[*]
Let $\mathcal{S}_\Integer$ be the symmetric group of $\Integer$, i.e.\ the set of all permutations of $\Integer$ under composition. Let $S$ be the element of $\mathcal{S}_\Integer$ defined by $S(n) = n+1$.

\begin{alist}
	
	\item[a] Show that the mapping $n \mapsto S^n$ is a bijection between $\Integer$ and $\{F \in \mathcal{S}_\Integer \colon F \circ S = S \circ F \}$.
	
	\item[b] Show that for any integers $m,n$, $n$ is divisible by $m$ iff $S^n$ commutes with every element of  $\mathcal{S}_\Integer$ with which $S^m$ commutes.
	
	\item[c] Let $\M$ be the structure $(\Integer, +, |, 1)$, where $\cdot | \cdot$ is the divisibility relation over $\Integer$. Show that $\M$ is strongly undecidable. 
	
	\item[d] Show that $\mathcal{S}_\Integer$ is strongly undecidable.

\end{alist}


\vspace{5ex}

\emph{Note: The last three exercises are from Shoenfield. He gives hints. I will post them on Sunday on bSpace. But if you want to use them right away, you know where to look.}

\end{document}

