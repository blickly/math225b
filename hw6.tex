\documentclass[12pt]{article}

%\usepackage{a4wide}
\usepackage[centertags]{amsmath}
\usepackage{amsthm}
\usepackage{amssymb}
\usepackage{amsfonts}
\usepackage{amscd,amsbsy}
\usepackage{txfonts}
\usepackage[T1]{fontenc} 

% \usepackage{eulervm}
\usepackage{eucal} 
%\usepackage{textcomp}

%\usepackage{mathptmx} 
\usepackage{enumerate}

\usepackage{pdfsync}

% Deutsche Texte
\usepackage[applemac]{inputenc}

% Seitenlayout
\pagestyle{empty}

\setlength{\textheight}{25cm}
\setlength{\textwidth}{16cm}

\setlength{\hoffset}{-1,5cm}
\setlength{\voffset}{-2,5cm}

\newcommand{\alabel}[1]{\mbox{(#1)}\hfill}
\newenvironment{alist}
        {\begin{list}{}%
                {\renewcommand{\makelabel}{\alabel}%
                \setlength{\parsep}{0.2cm}\setlength{\listparindent}{0cm}
                \setlength{\itemsep}{0.2cm}}}
        {\end{list}}

		\newcounter{aufgabenr}
\newcommand{\Aufg}[1][\mbox{}]{\subsubsection*{Problem \arabic{aufgabenr}#1}\stepcounter{aufgabenr}}
		\stepcounter{aufgabenr}
\newcommand{\problem}[2][\mbox{}]
{\subsubsection*{}
\fbox{\parbox{\textwidth}{
\vspace{-16pt}
\Aufg[#1]
#2}}
\medskip}

% Umgebung "Lösung"
\newenvironment{lsg}{\par\bigskip\noindent{\emph{Solution.\ }}\footnotesize}{\hfill $\blacksquare$}


% Zahlenmengen
\newcommand{\Nat}{\ensuremath{\mathbb{N}}}
\newcommand{\Integer}{\ensuremath{\mathbb{Z}}}
\newcommand{\Rat}{\ensuremath{\mathbb{Q}}}
\newcommand{\Real}{\ensuremath{\mathbb{R}}}
\newcommand{\Comp}[1]{\ensuremath{\overline{#1}}}
\newcommand{\Conj}[1]{\ensuremath{\overline{#1}}}

\newcommand{\Lang}{\ensuremath{\mathcal{L}}}

\newcommand{\LA}{\ensuremath{\mathcal{L}_A}}
\newcommand{\M}{\ensuremath{\mathcal{M}}}
\newcommand{\TA}{\ensuremath{\operatorname{Th}(\Nat)}}
\newcommand{\PA}{\ensuremath{\mathsf{PA}}}

\newcommand{\Term}[1]{\ensuremath{\underline{#1}}}

\newcommand{\Pow}[1]{\ensuremath{\mathcal{P}(#1)}}




\begin{document}

\begin{center}
{\LARGE Homework 6 for \textbf{MATH 225B}}

%\vspace{0,5cm}

{\large Author: Ben Lickly \\ Due: Thursday March 12}
\end{center}

\problem{
Show that there is a uniform procedure such that given $\emptyset^{(n)}$ for any $n$, we can calculate $n$, i.e.\ there exists a $z$ such that
\[
	\text{for all $n$ }, \Phi_z^{\emptyset^{(n)}}(0) = n.
\]
}

Let $e_0$ be the G\"odel numbering of any total Turing machine.

Let $f(n)$ be an oracle Turing machine program that does the following:

If $n = 0$, look at the tape at position $e_0$.
        If $tape[e_0] = 1$, return
        If $tape[e_0] = 0$, go into an infinite loop

Otherwise, find the G\"odel numbering of $f(n-1)$, call it $e'$.
Now look at the tape at position $e'$:
        If $tape[e'] = 1$, return
        If $tape[e] = 0$, go into an infinite loop

Now our overall uniform procedure is as follows:
\begin{verbatim}
for x in 0 to \infty
  let e_x be the godel number of f(x)
      if e_x is not on the tape, return x
\end{verbatim}

\problem{ 
\begin{alist}
	\item[a] Let $\{A_y\}_{y \in \Nat}$ be any sequence of sets. Define the \emph{infinite join} 
	\[
		\oplus \{A_y\}_{y \in \Nat} =  \{ \langle x,y\rangle \colon x \in A_y \}.
	\]
	Prove that $\oplus \{A_y\}_{y \in \Nat}$ is the \emph{uniform least upper bound} for $\{A_y\}$ in the sense that if there exist a set $C$ and a computable function $f$ such that $A_y = \Phi^C_{f(y)}$ for all $y$, then $\oplus \{A_y\} \leq_{T} C$.
	
	\item[b] Given a set $A$, let $A^{(\omega)}$ be defined as $\oplus \{A^{(n)}\}_{n \in \Nat}$. Show that for any $A$ there does not exist a set $B$ such that $A^{(\omega)} \equiv_m B'$.
\end{alist}
}


\problem{ 
\begin{alist}
	\item[a] Prove that if $B \leq_m A$ and $A = \emptyset^{(n)}$ for some $n \geq 1$, then $B \leq_1 A$.
	
	\item[b] Prove that if $B = \emptyset^{(n)}$ for some $n \geq 1$, and $B \leq_m A$, then $B \leq_1 A$.
\end{alist}
}

	
\problem{ 
Define the \emph{weak jump} as
\[
	H_A = \{ x \colon W_x \cap A \neq \emptyset \}.
\]
Prove that if $A$ is $\Pi_n$-complete, then $H_A$ is $\Sigma_{n+1}$-complete. Conclude that
\[
	H_{\operatorname{Inf}} = \{ x \colon \exists y \: (y \in W_x \: \wedge \: W_y \text{ is infinite}) \}
\]
is $\Sigma_3$-complete.
}


\problem[*]{
Recall that $2^\omega$ denotes the space of all infinite binary sequences. By identifying sets with their characteristic function, we can see $2^\omega$ as $\mathcal{P}(\Nat)$. The \emph{basic open sets} are given as
\[
	[\sigma] = \{ A \in 2^\omega \colon \sigma \subset A \}, \text{ where $\sigma$ is a finite binary string }
\] 
A set $X \subseteq 2^\omega$ has \emph{measure zero} if for any $\epsilon > 0$ there exists a set $\{\sigma_i \colon i \in \Nat\}$ of strings such that
\[
	X \subseteq \bigcup_{i \in \Nat} [\sigma_i] \quad \text{ and } \quad \sum_{i \in \Nat} 2^{-|\sigma_i|} < \epsilon.
\]
Show that if $A$ is not computable, then $\{B \colon B \geq_T A\}$ has measure zero.

[\emph{Remark:} You may want to consult some analysis text if you are not familiar with measure theory. In particular, look for the \emph{Lebesgue Density Theorem}.]
}

Assume we have an $A$ that is not computable.  This means that $K \le_T A$.

Now let $X = \{B \colon B \geq_T A \}$.  We want to show that $X$ has measure zero.

To do this, we will construct a set of strings,
$\{\sigma_i \colon i \in \Nat\}$, such that:
\[
	X \subseteq \bigcup_{i \in \Nat} [\sigma_i] \quad \text{ and } \quad \sum_{i \in \Nat} 2^{-|\sigma_i|} < \epsilon.
\]

\end{document}

