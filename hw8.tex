\documentclass[12pt]{article}

%\usepackage{a4wide}
\usepackage[centertags]{amsmath}
\usepackage{amsthm}
\usepackage{amssymb}
\usepackage{amsfonts}
\usepackage{amscd,amsbsy}
\usepackage{txfonts}
\usepackage[T1]{fontenc} 

% \usepackage{eulervm}
\usepackage{eucal} 
%\usepackage{textcomp}

%\usepackage{mathptmx} 
\usepackage{enumerate}


\usepackage{pdfsync}


% Deutsche Texte
\usepackage[applemac]{inputenc}

% Seitenlayout
\pagestyle{empty}

\setlength{\textheight}{25cm}
\setlength{\textwidth}{16cm}

\setlength{\hoffset}{-1,5cm}
\setlength{\voffset}{-2,5cm}

\newcommand{\alabel}[1]{\mbox{(#1)}\hfill}
\newenvironment{alist}
        {\begin{list}{}%
                {\renewcommand{\makelabel}{\alabel}%
                \setlength{\parsep}{0.2cm}\setlength{\listparindent}{0cm}
                \setlength{\itemsep}{0.2cm}}}
        {\end{list}}

		\newcounter{aufgabenr}
\newcommand{\Aufg}[1][\mbox{}]{\subsubsection*{Problem \arabic{aufgabenr}#1}\stepcounter{aufgabenr}}
		\stepcounter{aufgabenr}
\newcommand{\problem}[2][\mbox{}]
{\subsubsection*{}
\fbox{\parbox{\textwidth}{
\vspace{-16pt}
\Aufg[#1]
#2}}
\medskip}

% Umgebung "Lösung"
\newenvironment{lsg}{\par\bigskip\noindent{\emph{Solution.\ }}\footnotesize}{\hfill $\blacksquare$}


% Zahlenmengen
\newcommand{\Nat}{\ensuremath{\mathbb{N}}}
\newcommand{\Integer}{\ensuremath{\mathbb{Z}}}
\newcommand{\Rat}{\ensuremath{\mathbb{Q}}}
\newcommand{\Real}{\ensuremath{\mathbb{R}}}
\newcommand{\Comp}[1]{\ensuremath{\overline{#1}}}
\newcommand{\Conj}[1]{\ensuremath{\overline{#1}}}

\newcommand{\Lang}{\ensuremath{\mathcal{L}}}

\newcommand{\LA}{\ensuremath{\mathcal{L}_A}}
\newcommand{\M}{\ensuremath{\mathcal{M}}}
\newcommand{\TA}{\ensuremath{\operatorname{Th}(\Nat)}}
\newcommand{\PA}{\ensuremath{\mathsf{PA}}}

\newcommand{\GN}[1]{\ensuremath{\ulcorner #1 \urcorner}}

\newcommand{\Term}[1]{\ensuremath{\underline{#1}}}

\newcommand{\Pow}[1]{\ensuremath{\mathcal{P}(#1)}}


\newcommand{\gn}[1]{\ulcorner #1 \urcorner}

\begin{document}

\begin{center}
{\LARGE Homework 8 for \textbf{MATH 225B}}

%\vspace{0,5cm}

{\large Author: Ben Lickly \\ Due: Thursday April 9}
\end{center}


\problem{ 
Show that if $T$ is a decidable $\LA$-theory and $T' \supseteq T$ is a finite extension, then $T'$ is decidable.
}


\problem{
Show that, for all functions $f : \Nat^k \to \Nat$ (where $k\geq 1$), $f$ is representable 
in $\PA^-$ implies that $f$ is computable.
}


\problem{
Suppose $T$ is an $\LA$-theory and $\Delta$ is a c.e.\ set of sentences such that $T = \{\sigma \colon \Delta \models \sigma \}$. Show that $T$ is effectively (i.e.\ computably) axiomatizable.
}

\begin{proof}
In order to show that a theory $T$ is effectively axiomatizable, we need to
show a computable set of sentences $\Delta_{comp}$ such that
\[
T = \{ \sigma : \Delta_{comp} \models \sigma \}
\]

We cannot use $\Delta$ for this purpose since $\Delta$ is only computably
enumerable.  In order to construct this new set, we will use the fact that
a c.e.\ set with a total order that outputs its elements in order is
computable.
Using this fact, we will create a new set $\Delta_{comp}$ such that each of
its elements is greater than the last using the total order
$\varphi < \psi \Leftrightarrow \gn{\varphi} < \gn{\psi}$.
Our set will be 
\[
\Delta_{comp} = \{ \varphi_1 , \varphi_1 \wedge \varphi_2 , \dots \}
\]
where $\varphi_i$ is the $i$th sentence produced in the enumeration of
$\Delta$. Since taking conjunctions can only increase the G\"odel number of
a formula, this set will be strictly increasing.

Since $\bigwedge \Delta_{comp} = \bigwedge \Delta$, the theory formed by
the inductive closure of $\Delta_{comp}$ must also be the same as that
formed from $\Delta$.
\end{proof}




\problem{
An $\LA$-structure $\mathcal{A}$ is \emph{strongly undecidable} if every $\LA$-theory $T$ with $\mathcal{A} \models T$ is undecidable. 
\begin{itemize}
	\item[(a)] Show that $\Nat$ is strongly undecidable.
\end{itemize}
A theory $T$ is called \emph{hereditarily undecidable} if each subtheory of $T$ (i.e.\ any subset of $T$ closed under deduction) is undecidable.
\begin{itemize}
	\item[(b)] Show that $\mathcal{A}$ is strongly undecidable iff $\operatorname{Th}(\mathcal{A})$ is hereditarily undecidable.
	\item[(c)] Show that if $T$ has a strongly undecidable model, then $T$ is hereditarily undecidable.
\end{itemize}  
Conclude that every strongly undecidable theory is hereditarily undecidable.
}


\problem{
Prove the Lemma from class: Suppose $T$ allows representations and has a provability predicate $\theta$ that satisfies the conditions  
\begin{enumerate}[(P1)]
	\item $T \vdash \sigma$ implies $T \vdash \theta(\GN{\sigma})$
	\item $T \vdash [\theta(\GN{\sigma}) \: \wedge \:\theta(\GN{\sigma \to \tau})] \; \to \; \theta(\GN{\tau})$
	\item $T \vdash \theta(\GN{\sigma}) \; \to \; \theta(\GN{\theta(\GN{\sigma})})$
\end{enumerate}

Let $\sigma, \eta$ be $\LA$-sentences such that $T \vdash \sigma \: \leftrightarrow \: (\theta(\GN{\sigma}) \: \to \: \eta)$. Show that $$T \vdash \sigma \: \leftrightarrow \: (\theta(\GN{\eta}) \: \to \: \eta)$$.
}



\end{document}

