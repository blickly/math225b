\documentclass[12pt]{article}

%\usepackage{a4wide}
\usepackage[centertags]{amsmath}
\usepackage{amsthm}
\usepackage{amssymb}
\usepackage{amsfonts}
\usepackage{amscd,amsbsy}
\usepackage{txfonts}
\usepackage[T1]{fontenc} 

% \usepackage{eulervm}
\usepackage{eucal} 
%\usepackage{textcomp}

%\usepackage{mathptmx} 
\usepackage{enumerate}


\usepackage{pdfsync}


% Deutsche Texte
\usepackage[applemac]{inputenc}

% Seitenlayout
\pagestyle{empty}

\setlength{\textheight}{25cm}
\setlength{\textwidth}{16cm}

\setlength{\hoffset}{-1,5cm}
\setlength{\voffset}{-2,5cm}

\newcommand{\alabel}[1]{\mbox{(#1)}\hfill}
\newenvironment{alist}
        {\begin{list}{}%
                {\renewcommand{\makelabel}{\alabel}%
                \setlength{\parsep}{0.2cm}\setlength{\listparindent}{0cm}
                \setlength{\itemsep}{0.2cm}}}
        {\end{list}}

		\newcounter{aufgabenr}
\newcommand{\Aufg}[1][\mbox{}]{\subsubsection*{Problem \arabic{aufgabenr}#1}\stepcounter{aufgabenr}}
		\stepcounter{aufgabenr}
\newcommand{\problem}[2][\mbox{}]
{\subsubsection*{}
\fbox{\parbox{\textwidth}{
\vspace{-16pt}
\Aufg[#1]
#2}}
\medskip}


% Umgebung "Lösung"
\newenvironment{lsg}{\par\bigskip\noindent{\emph{Solution.\ }}\footnotesize}{\hfill $\blacksquare$}


% Zahlenmengen
\newcommand{\Nat}{\ensuremath{\mathbb{N}}}
\newcommand{\Integer}{\ensuremath{\mathbb{Z}}}
\newcommand{\Rat}{\ensuremath{\mathbb{Q}}}
\newcommand{\Real}{\ensuremath{\mathbb{R}}}
\newcommand{\Comp}{\ensuremath{\mathbb{C}}}
\newcommand{\Conj}[1]{\ensuremath{\overline{#1}}}

\newcommand{\Lang}{\ensuremath{\mathcal{L}}}

\newcommand{\LA}{\ensuremath{\mathcal{L}_A}}
\newcommand{\M}{\ensuremath{\mathcal{M}}}
\newcommand{\TA}{\ensuremath{\operatorname{Th}(\Nat)}}
\newcommand{\PA}{\ensuremath{\mathsf{PA}}}

\newcommand{\Term}[1]{\ensuremath{\underline{#1}}}
\newcommand{\Pow}[1]{\ensuremath{\mathcal{P}(#1)}}

\begin{document}

\begin{center}
{\LARGE Homework 3 for \textbf{MATH 225B}}

%\vspace{0,5cm}

{\large Author: Ben Lickly \\ Due: Thursday Feb 19}
\end{center}

\problem{ Prove that $K \equiv_1 K_0 \equiv_1 K_1$.}

\[
K, K_1 \le_1 K_0
\]

\[
K_0 \le_1 K \le_1 K_1 \le_1 K_0
\]

\problem{
Prove that the $m$-degree of $A\oplus B$ is the least upper bound 
(l.u.b.) of the $m$-degrees of $A$ and $B$, namely 
\begin{enumerate}[(a)]
\item $A \leq_m A \oplus B$ and $B \leq_m A \oplus B$; and
\item If $A \leq_m C$ and $B \leq_m C$ then $A\oplus B \leq_m C$.
\end{enumerate}
}

\problem{
Show that every infinite c.e.\ set has an infinite computable subset.
}

\problem[*]{
Prove that $\operatorname{Inf} \equiv_1 \operatorname{Tot} \equiv_1 \operatorname{Con}$
}

\iffalse
{\bf Prop}: 
If $A \le_m B$ and $B$ computable $\Rightarrow$ $A$ computable \\
\proof If $A \le_m B$ via $f$ then
\[
		\chi_A(x) = \chi_B(f(x))
\]

\bigskip

Ex:  $TOT = \{ x: \phi_x \text{ is total} \}$
	Claim:  $K \le_1 TOT$

\proof
Def:  \[\psi(x,y) = \begin{cases}  1, &\text{if }x \in K \\
          \uparrow, 	&o.w. \end{cases}\]
By the S-m-n Theorem there exists comp. 1-1 func. s.t.
	\[\phi_{f(x)}(y) = \psi(x,y)\]

Claim:  $K \le_1 TOT$ via $f$ \\
	If $x \in K \Rightarrow \psi(x,y) = 1$, f.all $y$ 
        $\Rightarrow \phi_{f(x)} \text{ is total} \Rightarrow f(x) \in TOT$ \\
        If $x \notin K \Rightarrow \psi(x,y) \uparrow, \text{f.all } y$ 
        $\Rightarrow \phi_{f(x)} \text{ is not total} \Rightarrow f(x) \notin TOT$

        \bigskip
        HW Blueprint:
\fi

\noindent
{ \bf Claim}: $ INF \le_1 CON $ \\
Define $\psi$:
\[ \psi(x,y) = \begin{cases} 0 &\exists z > y,\, \phi_x(z) \downarrow \\
  \uparrow &\text{otherwise} \end{cases}
\]
By the S-m-n Theorem there exists comp. 1-1 func. s.t.
	\[\phi_{f(x)}(y) = \psi(x,y)\]
Claim:  $INF \le_1 CON$ via $f$
\proof
\begin{eqnarray*}
	x \in INF &\Rightarrow& \psi(x,y) = 0, \text{ for all } y \\
	&\Rightarrow& \phi_{f(x)} \text{ is constant} \\
        &\Rightarrow& f(x) \in CON \\
        x \notin INF &\Rightarrow& (\exists y) \psi(x,y) \uparrow \\
	&\Rightarrow& \phi_{f(x)} \text{ is not constant} \\
        &\Rightarrow& f(x) \notin CON \\
\end{eqnarray*}

%%%%%%%%%%%%% TOT <= INF  %%%%%%%%%%%%%%%%%%%
\medskip
\noindent
{ \bf Claim}: $CON \le_1 TOT$ \\
Define $\psi$:
\[ \psi(x,y) = \begin{cases} 
  \phi_x(0)      &\text{if } \phi_x(y) \downarrow \\
  \uparrow      &\text{otherwise} \end{cases}
\]
By the S-m-n Theorem there exists comp. 1-1 func. s.t.
	\[\phi_{f(x)}(y) = \psi(x,y)\]
Claim:  $CON \le_1 TOT$ via $f$
\proof
\begin{eqnarray*}
	x \in CON &\Rightarrow& \psi(x,y) = \psi_x(0), \text{ for all } y \\
	&\Rightarrow& \phi_{f(x)} \text{ is total} \\
        &\Rightarrow& f(x) \in TOT \\
        x \notin CON 
	&\Rightarrow& (\exists y) \left(\phi_x(y) \ne \phi_x(0)
				  \text{ or } \phi_x(y)\uparrow \right) \\
        &\Rightarrow& (\exists y) \psi(x,y)\uparrow \\
	&\Rightarrow& \phi_{f(x)} \text{ is not total} \\
        &\Rightarrow& f(x) \notin TOT
\end{eqnarray*}

%%%%%%%%%%%%% TOT <= INF  %%%%%%%%%%%%%%%%%%%
\medskip
\noindent
{\bf Claim}: $ TOT \le_1 INF $ \\
Define $\psi$:
\[ \psi(x,y) = \begin{cases} 
  0             &\text{if } \forall z \le y,\, \phi_x(z) \downarrow \\
  \uparrow      &\text{otherwise} \end{cases}
\]
By the S-m-n Theorem there exists comp. 1-1 func. s.t.
	\[\phi_{f(x)}(y) = \psi(x,y)\]
Claim:  $TOT \le_1 INF$ via $f$
\proof
\begin{eqnarray*}
	x \in TOT &\Rightarrow& \psi(x,y) = 0, \text{ fpr all } y \\
	&\Rightarrow& W_{f(x)} \text{ is infinite} \\
        &\Rightarrow& f(x) \in INF \\
        x \notin TOT 
        &\Rightarrow& (\exists y)\psi(x,y)\uparrow \\
	&\Rightarrow& \phi_{f(x)} \uparrow, \text{ for all } z > y \\
	&\Rightarrow& W_{f(x)} \text{ is not infinite} \\
        &\Rightarrow& f(x) \notin INF
\end{eqnarray*}


\problem{
Define the \emph{busy beaver function} $\operatorname{BB}$ as
\[
	\operatorname{BB}(n) = \text{max number of $1$'s output by an $n$-state Turing machine on input $0$}
\]
Argue that $\operatorname{BB}$ is not computable. [Informal: How complicated is $\operatorname{BB}$? Can you ``reduce'' the halting problem to it?]
}



\end{document}

