\documentclass[12pt]{article}

%\usepackage{a4wide}
\usepackage[centertags]{amsmath}
\usepackage{amsthm}
\usepackage{amssymb}
\usepackage{amsfonts}
\usepackage{amscd,amsbsy}
\usepackage{txfonts}
\usepackage[T1]{fontenc} 

% \usepackage{eulervm}
\usepackage{eucal} 
%\usepackage{textcomp}

%\usepackage{mathptmx} 
\usepackage{enumerate}


\usepackage{pdfsync}


% Deutsche Texte
\usepackage[applemac]{inputenc}

% Seitenlayout
\pagestyle{empty}

\setlength{\textheight}{25cm}
\setlength{\textwidth}{16cm}

\setlength{\hoffset}{-1,5cm}
\setlength{\voffset}{-2,5cm}

\newcommand{\alabel}[1]{\mbox{(#1)}\hfill}
\newenvironment{alist}
        {\begin{list}{}%
                {\renewcommand{\makelabel}{\alabel}%
                \setlength{\parsep}{0.2cm}\setlength{\listparindent}{0cm}
                \setlength{\itemsep}{0.2cm}}}
        {\end{list}}

		\newcounter{aufgabenr}
\newcommand{\Aufg}[1][\mbox{}]{\subsubsection*{Problem \arabic{aufgabenr}#1}\stepcounter{aufgabenr}}
		\stepcounter{aufgabenr}
\newcommand{\problem}[2][\mbox{}]
{\subsubsection*{}
\fbox{\parbox{\textwidth}{
\vspace{-16pt}
\Aufg[#1]
#2}}
\medskip}


% Umgebung "Loesung"
\newenvironment{lsg}{\par\bigskip\noindent{\emph{Solution.\ }}\footnotesize}{\hfill $\blacksquare$}


% Zahlenmengen
\newcommand{\Nat}{\ensuremath{\mathbb{N}}}
\newcommand{\Integer}{\ensuremath{\mathbb{Z}}}
\newcommand{\Rat}{\ensuremath{\mathbb{Q}}}
\newcommand{\Real}{\ensuremath{\mathbb{R}}}
\newcommand{\Comp}[1]{\ensuremath{\overline{#1}}}
\newcommand{\Conj}[1]{\ensuremath{\overline{#1}}}

\newcommand{\Lang}{\ensuremath{\mathcal{L}}}

\newcommand{\LA}{\ensuremath{\mathcal{L}_A}}
\newcommand{\M}{\ensuremath{\mathcal{M}}}
\newcommand{\TA}{\ensuremath{\operatorname{Th}(\Nat)}}
\newcommand{\PA}{\ensuremath{\mathsf{PA}}}

\newcommand{\Term}[1]{\ensuremath{\underline{#1}}}

\newcommand{\Pow}[1]{\ensuremath{\mathcal{P}(#1)}}




\begin{document}

\begin{center}
{\LARGE Homework 5 for \textbf{MATH 225B}}

%\vspace{0,5cm}

{\large Author: Ben Lickly \\ Due: Thursday March 5}
\end{center}

\problem{
Use the Recursion Theorem to prove that $K$ is not an index set.
}
%If K were computable, so would be
%    \psi(e,x) = {   \phi_e(x)       if e \ne x
%                    \phi_e(e)+1     if e = x, e \in V
%                    0               if e = x, e \not\in V
%               Let f be s.t. \psi(e,x) = \phi_{f(e)}(x)
%               Then f.all e, \phi_{f(e)} \ne \phi_e            by Rec Thm

\begin{proof}
Assume that $K$ were an index set.
This means that for all $x,y$
\[
x \in K \wedge (\phi_x=\phi_y) \Rightarrow y \in K
\]

Let \[
    \psi(e,x) = \begin{cases}
	\uparrow	&\text{if } e \ne x \\
	0		&\text{if } e = x \\
\end{cases}
\]
 Let $f$ be such that $\psi(e,x) = \phi_{f(e)}(x)$.

By the Recursion Theorem, there exists an $e$ such that
$\phi_e = \phi_{f(e)}$.

Since $\phi_e(e) \downarrow$, $e \in K$.
By the padding lemma, we can find a new index $e'$ for this same function.
Even though we still have $\phi_{e'}(e) \downarrow$, 
when applied to it's own index we get $\phi_{e'}(e') \uparrow$.
Thus $e' \not\in K$. 
We now have
\[
e \in K \wedge (\phi_e=\phi_{e'}) \not\Rightarrow e' \in K
\]

Thus $K$ can not be an index set.
\end{proof}

%Let \[
%    \psi(e,x) = \begin{cases}
%	\phi_e(x)	&\text{if } e \ne x \\
%	\phi_e(e)+1	&\text{if } e = x, e \in K \\
%	0		&\text{if } e = x, e \not\in K \\
%\end{cases}
%\]
%               Let $f$ be s.t. $\psi(e,x) = \phi_{f(e)}(x)$.

%This means that for any computable $f$, there exists an $e$ such that 
%$\phi_{f(e)} = \phi_e$.

\problem{ 
\begin{enumerate}[(a)]
\item Show that there exists a computable function $f$ whose set of fixed points is not c.e.

\item Show that if the set of fixed-points for $f$ is computable, then 
  \begin{equation*}
    \{ \varphi_x \colon \varphi_x = \varphi_{f(x)} \}
  \end{equation*}
includes all partial computable functions.
\end{enumerate}
}

\problem{ 
Let $g$ be any computable function. Show that there exists an $n$ such that
\begin{equation*}
  W_n \text{ computable } \quad \text{ and } \quad \mu y [W_y = \overline{W}_n]  > g(n).
\end{equation*}
}

\begin{proof}
Let \[
    \psi(n,x) = \begin{cases}
	1	        &\text{if } (\exists e \le g(n)), 
                              \text{the first output of $\phi_e$ is $x$,} \\
        \uparrow	&\text{otherwise.} \\
\end{cases}
\]
 Let $f$ be such that $\psi(n,x) = \phi_{f(n)}(x)$.

By the Recursion Theorem, there exists an $n$ such that
$\phi_n = \phi_{f(n)}$.

Since $g$ is computable, we know that $g(x) \downarrow$ for any $x$.
As a result, $W_n$ is also finite with no more than $g(n)$ elements. 
Since any finite set is computable, $W_n$ must be computable.

By construction, for any $y \le g(n)$, if $W_y \ne \emptyset$, then there is
an element of $W_y$ in $W_n$, namely the first output of $\phi_y$.
Thus $W_y \ne \Conj{W_n}$.
%$(\exists x_y \in W_y) x_y \in W_n$.
If $W_y = \emptyset$, then since $\Conj{W_n}$ is cofinite, clearly 
$W_y \ne \Conj{W_n}$.
\end{proof}

\problem{
Show that there exists no computable function $f(x,s)$ such that for all $x$, $\hat{f}(x) = \lim_s f(x,s)$ exists (i.e.\ $f(x,s)$ is eventually stable), and $\hat{f}$ is the characteristic function of \textsc{Tot}.
}

\problem{
State and prove relativized versions of the S-m-n Theorem and the Recursion Theorem.
}
%Relativized Enumeration Theorem:
%There exist $z \in N$ such that for all $A \subseteq N$, $x, y \in N$
%\[\Phi^A_z(x,y) = \Phi^A_x(y)\]

\newtheorem*{relsmn}{Relativized S-m-n Theorem}
\begin{relsmn}
For every $m,n \ge 1$, there exists a 1-1 comp function $S_n^m$ of $m+1$ variables such that f.all $A\subseteq \Nat$ and $x,y_1,\ldots,y_m, z_1, \ldots, z_n \in \Nat$:
\[\Phi^A_{S_n^m(x,y_1,\ldots,y_m)}(z_1, \ldots, z_n)
= \Phi^A_{x}(y_1,\ldots,y_m,z_1,\ldots,z_n)\]
		($y$'s are params, $z$'s are vars)
\end{relsmn}
%
\begin{proof}
Without loss of generality, let $m=n=1$ \\
In order to simulate $\tilde{P}_{S_1^1(x,y)}$
on work input $z$ and oracle input $A$, we must do the following:
\begin{itemize}
  \item
    obtain $\tilde{P}_x$ (the $x$th 2-variable OTM),
  \item
    apply $\tilde{P}_x$ to work input $(y,z)$ and oracle input $A$.
\end{itemize}
Finding index for given program is uniformly effective in $x,y$, this defines $S_1^1$

If $S_1^1$ is not 1-1, we can use the relativized Padding Lemma to define new function $\tilde{S}_1^1$ strictly increasing in $<x,y>$
such that $\Phi^A_{\tilde{S}_1^1(x,y)} = \Phi^A_{S_1^1(x,y)}$
\end{proof}


\newtheorem*{relrec}{Relativized Recursion Theorem}
\begin{relrec}
For every computable $f: \Nat \rightarrow \Nat$, 
and arbitrary $A \subseteq \Nat$,
there exists an $n$ such that $\Phi^A_n = \Phi^A_{f(n)}$
	($n$ is a semantical fixed-point of $f$)
\end{relrec}
%
\begin{proof}
Using the Relativized S-m-n Theorem, define a 1-1 computable $d$ such that:
\[
\Phi^A_{d(u)}(z) = \begin{cases}
  \Phi^A_{\Phi^A_u(u)}(z)		&\text{if } \Phi^A_u(u) \downarrow, \\
  \uparrow			&\text{otherwise.}
\end{cases}
\]
Let $v$ be such that $\Phi^A_v = f \circ d$ \\
{\bf Claim}:  
\begin{align*}
  & n = d(v) \text{ is a fixed point for } f \\
  \Rightarrow& f \circ d = \Phi^A_v \text{ is total} \\
  \Rightarrow& \Phi^A_{d(v)} = \Phi^A_{\Phi^A_v(v)}
\end{align*}
Now: \[
\Phi^A_n = \Phi^A_{d(v)} = \Phi^A_{\Phi^A_v(v)} = \Phi^A_{(f \circ d)(v)} = \Phi^A_{f(d(v))} = \Phi^A_{f(n)} 
\]
\end{proof}

\end{document}

