\documentclass[12pt]{article}

%\usepackage{a4wide}
\usepackage[centertags]{amsmath}
\usepackage{amsthm}
\usepackage{amssymb}
\usepackage{amsfonts}
\usepackage{amscd,amsbsy}
\usepackage{txfonts}
\usepackage[T1]{fontenc} 

% \usepackage{eulervm}
\usepackage{eucal} 
%\usepackage{textcomp}

%\usepackage{mathptmx} 
\usepackage{enumerate}

\usepackage{pdfsync}

% Deutsche Texte
\usepackage[applemac]{inputenc}

% Seitenlayout
\pagestyle{empty}

\setlength{\textheight}{25cm}
\setlength{\textwidth}{16cm}

\setlength{\hoffset}{-1,5cm}
\setlength{\voffset}{-2,5cm}

\newcommand{\alabel}[1]{\mbox{(#1)}\hfill}
\newenvironment{alist}
        {\begin{list}{}%
                {\renewcommand{\makelabel}{\alabel}%
                \setlength{\parsep}{0.2cm}\setlength{\listparindent}{0cm}
                \setlength{\itemsep}{0.2cm}}}
        {\end{list}}

		\newcounter{aufgabenr}
\newcommand{\Aufg}[1][\mbox{}]{\subsubsection*{Problem \arabic{aufgabenr}#1}\stepcounter{aufgabenr}}
		\stepcounter{aufgabenr}
\newcommand{\problem}[2][\mbox{}]
{\subsubsection*{}
\fbox{\parbox{\textwidth}{
\vspace{-16pt}
\Aufg[#1]
#2}}
\medskip}

% Umgebung "Lösung"
\newenvironment{lsg}{\par\bigskip\noindent{\emph{Solution.\ }}\footnotesize}{\hfill $\blacksquare$}


% Zahlenmengen
\newcommand{\Nat}{\ensuremath{\mathbb{N}}}
\newcommand{\Integer}{\ensuremath{\mathbb{Z}}}
\newcommand{\Rat}{\ensuremath{\mathbb{Q}}}
\newcommand{\Real}{\ensuremath{\mathbb{R}}}
\newcommand{\Comp}[1]{\ensuremath{\overline{#1}}}
\newcommand{\Conj}[1]{\ensuremath{\overline{#1}}}

\newcommand{\Lang}{\ensuremath{\mathcal{L}}}

\newcommand{\LA}{\ensuremath{\mathcal{L}_A}}
\newcommand{\M}{\ensuremath{\mathcal{M}}}
\newcommand{\TA}{\ensuremath{\operatorname{Th}(\Nat)}}
\newcommand{\PA}{\ensuremath{\mathsf{PA}}}

\newcommand{\Term}[1]{\ensuremath{\underline{#1}}}

\newcommand{\Pow}[1]{\ensuremath{\mathcal{P}(#1)}}




\begin{document}

\begin{center}
{\LARGE Homework 6 for \textbf{MATH 225B}}

%\vspace{0,5cm}

{\large Author: Ben Lickly \\ Due: Thursday March 12}
\end{center}

\problem{
Show that there is a uniform procedure such that given $\emptyset^{(n)}$ for any $n$, we can calculate $n$, i.e.\ there exists a $z$ such that
\[
	\text{for all $n$ }, \Phi_z^{\emptyset^{(n)}}(0) = n.
\]
}

%Let $e_0$ be the G\"odel numbering of an Oracle Turing machine that
%halts immediately without looking at either of its tapes.

Let $\Psi_p^A(n,x)$ be an Oracle Turing machine program, as defined below.
Since $p$ is fixed, by the S-m-n theorem, there exists a computable function
$f$, such that $\Psi_{f(n)}^A(x) = \Psi_p^A(n,x)$.
Since they are equivalent, we choose to define
the latter function for simplicity.

We define $\Psi_{f(n)}^A(x)$ as follows:
%We will define $\Psi_p^A(n,x)$ as follows:
If $n = 0$, we define a simple function that is total (and thus $f(0) \in K$).
\[
\Psi_{f(0)}^A(x) = 0
\]
%as follows:
%        If $tape[e_0] = 1$, return
%        If $tape[e_0] = 0$, go into an infinite loop
Otherwise, we will use the G\"odel numbering of $\Psi_{f(n-1)}^A(n-1,x)$:
\[
\Psi_{f(n)}^A(x) = \begin{cases}
                0               &\text{ if $A[f(n-1)] = 1$} \\
                \uparrow        &\text{ if $A[f(n-1)] = 0$} \\
\end{cases}
\]

%Otherwise, we will use the G\"odel numbering of $f(n-1)$, call it $e'$.
%Now look at the tape at position $e'$:
%        If $tape[e'] = 1$, return
%        If $tape[e] = 0$, go into an infinite loop

Now our overall uniform procedure is as follows:
\[
\Phi_z^{A}(y) = (\mu x) ( f(x) \not\in A )
\]

\problem{ 
\begin{alist}
	\item[a] Let $\{A_y\}_{y \in \Nat}$ be any sequence of sets. Define the \emph{infinite join} 
	\[
		\oplus \{A_y\}_{y \in \Nat} =  \{ \langle x,y\rangle \colon x \in A_y \}.
	\]
	Prove that $\oplus \{A_y\}_{y \in \Nat}$ is the \emph{uniform least upper bound} for $\{A_y\}$ in the sense that if there exist a set $C$ and a computable function $f$ such that $A_y = \Phi^C_{f(y)}$ for all $y$, then $\oplus \{A_y\} \leq_{T} C$.
	
	\item[b] Given a set $A$, let $A^{(\omega)}$ be defined as $\oplus \{A^{(n)}\}_{n \in \Nat}$. Show that for any $A$ there does not exist a set $B$ such that $A^{(\omega)} \equiv_m B'$.
\end{alist}
}
\begin{alist}
 \item[a] We can define this Turing reduction in a straightforward manner,
   simply using the following reduction:
   \[
   \Phi_e^A(\langle x, y\rangle) = \Phi_{f(y)}^A(x) 
   \]
   In the case that the oracle tape is $C$, 
   we have $\Phi_e^C = \oplus \{A_y\}$, just as we wanted.
 \item[b]
   Using the Jump Theorem, we can translate the $\equiv_m$ into a $\equiv_T$.
   %
   Now, assume that for some set $A$, there did exist a $B$ such that
   $A^{(\omega)} \equiv_T B'$.
   This would mean that $A^{(n)} \le_T B$, since
   $A^{(n+1)} \le_T A^{(\omega)} \equiv_T B'$.
   Since this is true for all $n$, this means that $B$ is a uniform upper
   bound for $A^{(n)}$, but it is distinct from $A^{(\omega)}$.
   This contradicts our result from part (a) that $A^{(\omega)}$ was the
   uniform least upper bound.
\end{alist}

\problem{ 
\begin{alist}
	\item[a] Prove that if $B \leq_m A$ and $A = \emptyset^{(n)}$ for some $n \geq 1$, then $B \leq_1 A$.
	
	\item[b] Prove that if $B = \emptyset^{(n)}$ for some $n \geq 1$, and $B \leq_m A$, then $B \leq_1 A$.
\end{alist}
}

	
\problem{ 
Define the \emph{weak jump} as
\[
	H_A = \{ x \colon W_x \cap A \neq \emptyset \}.
\]
Prove that if $A$ is $\Pi_n$-complete, then $H_A$ is $\Sigma_{n+1}$-complete. Conclude that
\[
	H_{\operatorname{Inf}} = \{ x \colon \exists y \: (y \in W_x \: \wedge \: W_y \text{ is infinite}) \}
\]
is $\Sigma_3$-complete.
}


\problem[*]{
Recall that $2^\omega$ denotes the space of all infinite binary sequences. By identifying sets with their characteristic function, we can see $2^\omega$ as $\mathcal{P}(\Nat)$. The \emph{basic open sets} are given as
\[
	[\sigma] = \{ A \in 2^\omega \colon \sigma \subset A \}, \text{ where $\sigma$ is a finite binary string }
\] 
A set $X \subseteq 2^\omega$ has \emph{measure zero} if for any $\epsilon > 0$ there exists a set $\{\sigma_i \colon i \in \Nat\}$ of strings such that
\[
	X \subseteq \bigcup_{i \in \Nat} [\sigma_i] \quad \text{ and } \quad \sum_{i \in \Nat} 2^{-|\sigma_i|} < \epsilon.
\]
Show that if $A$ is not computable, then $\{B \colon B \geq_T A\}$ has measure zero.

[\emph{Remark:} You may want to consult some analysis text if you are not familiar with measure theory. In particular, look for the \emph{Lebesgue Density Theorem}.]
}

%% First try
%Assume we have an $A$ that is not computable.  This means that $K \le_T A$.

%Now let $X = \{B \colon B \geq_T A \}$.  We want to show that $X$ has measure zero.

%To do this, we will construct a set of strings,
%$\{\sigma_i \colon i \in \Nat\}$, such that:
%\[
%	X \subseteq \bigcup_{i \in \Nat} [\sigma_i] \quad \text{ and } \quad \sum_{i \in \Nat} 2^{-|\sigma_i|} < \epsilon.
%\]

% From bspace
%Assume the upper cone $C$ of $X$ does not have measure zero. Since countable unions of measure zero sets have measure zero, it suffices to consider the upper cone with respect to a single reduction $Phi_e$. By the Lebesgue Density Theorem, there exists a $Y$ such that $C$ has density 1 along $Y$, i.e. eventually almost every $Z$ close to $Y$ (in the Cantor space topology) will compute $X$. That means that there exists some finite initial segment sigma above which 3/4 of all reals compute $X$. Now to compute $X(n)$, wait till more than half of the oracles extending sigma converge for $Phi_e$ and yield the same value. This value is $X(n)$.

Start with an $A$ that is not computable.  This means that $K \le_T A$.

Now assume the upper cone $C$ of $X$ does not have measure zero. 
Since countable unions of measure zero sets have measure zero, it suffices to consider the upper cone with respect to a single reduction $\varphi_e$. 
By the Lebesgue Density Theorem, there exists a $Y$ such that $C$ has density 1 along $Y$, i.e. eventually almost every $Z$ close to $Y$ (in the Cantor space topology) will compute $X$. 

That means that there exists some finite initial segment sigma above which 3/4 of all reals compute $X$. Now to compute $X(n)$, wait until more than half of the oracles extending sigma converge for $\varphi_e$ and yield the same value. This value is $X(n)$.

\end{document}

