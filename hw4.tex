\documentclass[12pt]{article}

%\usepackage{a4wide}
\usepackage[centertags]{amsmath}
\usepackage{amsthm}
\usepackage{amssymb}
\usepackage{amsfonts}
\usepackage{amscd,amsbsy}
%\usepackage{txfonts}
\usepackage[T1]{fontenc} 

% \usepackage{eulervm}
\usepackage{eucal} 
%\usepackage{textcomp}

%\usepackage{mathptmx} 
\usepackage{enumerate}


\usepackage{pdfsync}


% Deutsche Texte
\usepackage[applemac]{inputenc}

% Seitenlayout
\pagestyle{empty}

\setlength{\textheight}{25cm}
\setlength{\textwidth}{16cm}

\setlength{\hoffset}{-1,5cm}
\setlength{\voffset}{-2,5cm}

\newcommand{\alabel}[1]{\mbox{(#1)}\hfill}
\newenvironment{alist}
        {\begin{list}{}%
                {\renewcommand{\makelabel}{\alabel}%
                \setlength{\parsep}{0.2cm}\setlength{\listparindent}{0cm}
                \setlength{\itemsep}{0.2cm}}}
        {\end{list}}

		\newcounter{aufgabenr}
\newcommand{\Aufg}[1][\mbox{}]{\subsubsection*{Problem \arabic{aufgabenr}#1}\stepcounter{aufgabenr}}
		\stepcounter{aufgabenr}
\newcommand{\problem}[2][\mbox{}]
{\subsubsection*{}
\fbox{\parbox{\textwidth}{
\vspace{-16pt}
\Aufg[#1]
#2}}
\medskip}


% Umgebung "Lösung"
\newenvironment{lsg}{\par\bigskip\noindent{\emph{Solution.\ }}\footnotesize}{\hfill $\blacksquare$}


% Zahlenmengen
\newcommand{\Nat}{\ensuremath{\mathbb{N}}}
\newcommand{\Integer}{\ensuremath{\mathbb{Z}}}
\newcommand{\Rat}{\ensuremath{\mathbb{Q}}}
\newcommand{\Real}{\ensuremath{\mathbb{R}}}
\newcommand{\Comp}[1]{\ensuremath{\overline{#1}}}
\newcommand{\Conj}[1]{\ensuremath{\overline{#1}}}

\newcommand{\Lang}{\ensuremath{\mathcal{L}}}

\newcommand{\LA}{\ensuremath{\mathcal{L}_A}}
\newcommand{\M}{\ensuremath{\mathcal{M}}}
\newcommand{\TA}{\ensuremath{\operatorname{Th}(\Nat)}}
\newcommand{\PA}{\ensuremath{\mathsf{PA}}}

\newcommand{\Term}[1]{\ensuremath{\underline{#1}}}

\newcommand{\Pow}[1]{\ensuremath{\mathcal{P}(#1)}}




\begin{document}

\begin{center}
{\LARGE Homework 4 for \textbf{MATH 225B}}

%\vspace{0,5cm}

{\large Author: Ben Lickly \\ Due: Thursday Feb 26}
\end{center}

\problem{ Prove that $A \leq_m B$ and $B$ c.e.\ imply that $A$ is c.e. Use this to show that $\operatorname{Fin}$ and $\operatorname{Cof}$ are not c.e. }



\problem{ Show that there are infinitely many sets that are c.e.\ but not computable.}



\problem{
A function $f: \Nat \to \Nat$ is \emph{increasing} if $f(x) < f(x+1)$ for all $x$. Show that an infinite set $A$ is computable iff it is the range of an increasing computable function.

Use this to show that there exists a computable $f$ such that $\{W_{f(n)}\}$ consists precisely of the computable sets.
}


\problem[*]{
Show that $\Comp{K}$ contains an infinite c.e.\ subset. 
Generalize this to: If $A$ is $m$-complete, then $\Comp{A}$ contains an infinite c.e.\ subset.
}

\emph{Lemma}:  
We can effectively compute the G\"odel numbering, $e$, of any partial 
function for which $W_e$ is finite. 
i.e.
\[
f(x) = \begin{cases} 0 &\text{if } x \in S, \\
  \uparrow &\text{otherwise.} \end{cases}
  \]
  For a finite set $S$.

  {\bf Proof}: For each of the numbers in $S$, we can hardcode necessary
  states in a Turing Machine to output each element.  Since there are
  a finite number of elements, this takes a finite number of steps.
  After doing this, we can simply make the Turing machine go into an infinite
  loop.

  \bigskip
{\bf Overall Proof}:
In the first case, we take $S_0 = \emptyset$, and we can find the 
G\"odel number, $e_0$, for the function $f_0(x) = \uparrow$.
Clearly this function does not halt on any input, and so $e_0 \not\in K$ or 
$e_0 \in \Comp{K}$.

For each following step, let $S_k = \{ e_0, \ldots, e_{k-1} \}$.
We can find the G\"odel number, $e_k$ for 
$f_k$,
and we can be sure that $e_k \not\in S_k$ since the function it computes
is different from and $e_i$ with $i < k$.
Thus we know that we can add $e_k$ to $S_{k+1}$ and still have
$S_{k+1} \subseteq \Comp{K}$.

We can continue this process indefinitely to produce an infinite c.e.
set $S_{\infty} = \bigcup_{k \in \Nat} S_k$, which is a subset of $\Comp{K}$.

%If we have a Turing machine that produces a set $s_1, \ldots, s_{k-1}$, and a new element $

\bigskip

For the general case, we can extend this proof to work for any set $\Comp{A}$, 
where $A$ is $m$-complete.
This means that $K \le_m A$, or there exists a 
function $f : \Nat \rightarrow \Nat$, such that 
$x \in K \Leftrightarrow x \in A$.

Now, at each step of our proof, instead of just adding $e_k$ to $S_{k+1}$, 
we add $f(e_k)$.  The rest of the proof remains the same.

\problem{
A set $A$ is \emph{self-dual} if $A \leq_m \Comp{A}$. (For example, $A = B \oplus \Comp{B}$ is self-dual for any $B$). Use the Recursion Theorem to show that no index set can be self-dual and give a new proof of Rice's Theorem.
}

Let $A$ be an index set.
Since $A$ is an index set, there must be a $\phi$, such that:
\[
\forall x,y\ 	[x \in A \wedge \phi_x=\phi_y \Rightarrow y \in A]
\]

Assume that $A$ is also \emph{self-dual}, or $A \le_m \Comp{A}$.
This means that there exists a computable function $f: \Nat \rightarrow \Nat$, 
such that $x \in A \Leftrightarrow f(x) \in \Comp{A}$.
In other words, $x \in A \Leftrightarrow f(x) \not\in A$.
Thus, we have:
for all $i$, $\phi_i \ne \phi_{f(i)}$.

By the recursion theorem, however, we know that there exists some index $i$, such that 
\[
        \phi_i = \phi_{f(i)}
\]
But this is a contradiction.

Thus, no index set may be self-dual.

\end{document}

