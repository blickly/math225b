\documentclass[12pt]{article}

%\usepackage{a4wide}
\usepackage[centertags]{amsmath}
\usepackage{amsthm}
\usepackage{amssymb}
\usepackage{amsfonts}
\usepackage{amscd,amsbsy}
\usepackage{txfonts}
\usepackage[T1]{fontenc} 

% \usepackage{eulervm}
\usepackage{eucal} 
%\usepackage{textcomp}

%\usepackage{mathptmx} 
\usepackage{enumerate}


\usepackage{pdfsync}


% Deutsche Texte
\usepackage[applemac]{inputenc}

% Seitenlayout
\pagestyle{empty}

\setlength{\textheight}{25cm}
\setlength{\textwidth}{16cm}

\setlength{\hoffset}{-1,5cm}
\setlength{\voffset}{-2,5cm}

\newcommand{\alabel}[1]{\mbox{(#1)}\hfill}
\newenvironment{alist}
        {\begin{list}{}%
                {\renewcommand{\makelabel}{\alabel}%
                \setlength{\parsep}{0.2cm}\setlength{\listparindent}{0cm}
                \setlength{\itemsep}{0.2cm}}}
        {\end{list}}

		\newcounter{aufgabenr}
\newcommand{\Aufg}[1][\mbox{}]{\subsubsection*{Problem \arabic{aufgabenr}#1}\stepcounter{aufgabenr}}
		\stepcounter{aufgabenr}
\newcommand{\problem}[2][\mbox{}]
{\subsubsection*{}
\fbox{\parbox{\textwidth}{
\vspace{-16pt}
\Aufg[#1]
#2}}
\medskip}


% Umgebung "Loesung"
\newenvironment{lsg}{\par\bigskip\noindent{\emph{Solution.\ }}\footnotesize}{\hfill $\blacksquare$}


% Zahlenmengen
\newcommand{\Nat}{\ensuremath{\mathbb{N}}}
\newcommand{\Integer}{\ensuremath{\mathbb{Z}}}
\newcommand{\Rat}{\ensuremath{\mathbb{Q}}}
\newcommand{\Real}{\ensuremath{\mathbb{R}}}
\newcommand{\Comp}[1]{\ensuremath{\overline{#1}}}
\newcommand{\Conj}[1]{\ensuremath{\overline{#1}}}

\newcommand{\Lang}{\ensuremath{\mathcal{L}}}

\newcommand{\LA}{\ensuremath{\mathcal{L}_A}}
\newcommand{\M}{\ensuremath{\mathcal{M}}}
\newcommand{\TA}{\ensuremath{\operatorname{Th}(\Nat)}}
\newcommand{\PA}{\ensuremath{\mathsf{PA}}}

\newcommand{\Term}[1]{\ensuremath{\underline{#1}}}

\newcommand{\Pow}[1]{\ensuremath{\mathcal{P}(#1)}}




\begin{document}

\begin{center}
{\LARGE Homework 5 for \textbf{MATH 225B}}

%\vspace{0,5cm}

{\large Author: Ben Lickly \\ Due: Thursday March 5}
\end{center}

\problem{
Use the Recursion Theorem to prove that $K$ is not an index set.
}

\problem{ 
\begin{enumerate}[(a)]
\item Show that there exists a computable function $f$ whose set of fixed points is not c.e.

\item Show that if the set of fixed-points for $f$ is computable, then 
  \begin{equation*}
    \{ \varphi_x \colon \varphi_x = \varphi_{f(x)} \}
  \end{equation*}
includes all partial computable functions.
\end{enumerate}
}

\problem{ 
Let $g$ be any computable function. Show that there exists an $n$ such that
\begin{equation*}
  W_n \text{ computable } \quad \text{ and } \quad \mu y [W_y = \overline{W}_n]  > g(n).
\end{equation*}
}

\problem{
Show that there exists no computable function $f(x,s)$ such that for all $x$, $\hat{f}(x) = \lim_s f(x,s)$ exists (i.e.\ $f(x,s)$ is eventually stable), and $\hat{f}$ is the characteristic function of \textsc{Tot}.
}

\problem{
State and prove relativized versions of the S-m-n Theorem and the Recursion Theorem.
}
%Relativized Enumeration Theorem:
%There exist $z \in N$ such that for all $A \subseteq N$, $x, y \in N$
%\[\Phi^A_z(x,y) = \Phi^A_x(y)\]

S-m-n Theorem:
For every $m,n \ge 1$, there exists a 1-1 comp function $S_n^m$ of $m+1$ variables such that f.all $A\subseteq \Nat$ and $x,y_1,\ldots,y_m, z_1, \ldots, z_n \in \Nat$:
\[\Phi^A_{S_n^m(x,y_1,\ldots,y_m)}(z_1, \ldots, z_n)
= \Phi^A_{x}(y_1,\ldots,y_m,z_1,\ldots,z_n)\]
		($y$'s are params, $z$'s are vars)

\begin{proof}
Wlog $m=n=1$ \\
In order to simulate $\tilde{P}_{S_1^1(x,y)}$
on work input $z$ and oracle input $A$, we must do the following:
\begin{itemize}
  \item
    obtain $\tilde{P}_x$ (the $x$th 2-variable OTM),
  \item
    apply $\tilde{P}_x$ to work input $(y,z)$ and oracle input $A$.
\end{itemize}
Finding index for given program is uniformly effective in $x,y$, this defines $S_1^1$

If $S_1^1$ is not 1-1, we can use the relativized Padding Lemma to define new function $\tilde{S}_1^1$ strictly increasing in $<x,y>$
such that $\Phi^A_{\tilde{S}_1^1(x,y)} = \Phi^A_{S_1^1(x,y)}$
\end{proof}

\end{document}

