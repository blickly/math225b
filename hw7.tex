\documentclass[12pt]{article}

%\usepackage{a4wide}
\usepackage[centertags]{amsmath}
\usepackage{amsthm}
\usepackage{amssymb}
\usepackage{amsfonts}
\usepackage{amscd,amsbsy}
\usepackage{txfonts}
\usepackage[T1]{fontenc} 
\usepackage{undertilde}

% \usepackage{eulervm}
\usepackage{eucal} 
%\usepackage{textcomp}

%\usepackage{mathptmx} 
\usepackage{enumerate}


\usepackage{pdfsync}


% Deutsche Texte
\usepackage[applemac]{inputenc}

% Seitenlayout
\pagestyle{empty}

\setlength{\textheight}{25cm}
\setlength{\textwidth}{16cm}

\setlength{\hoffset}{-1,5cm}
\setlength{\voffset}{-2,5cm}

\newcommand{\alabel}[1]{\mbox{(#1)}\hfill}
\newenvironment{alist}
        {\begin{list}{}%
                {\renewcommand{\makelabel}{\alabel}%
                \setlength{\parsep}{0.2cm}\setlength{\listparindent}{0cm}
                \setlength{\itemsep}{0.2cm}}}
        {\end{list}}

		\newcounter{aufgabenr}
\newcommand{\Aufg}[1][\mbox{}]{\subsubsection*{Problem \arabic{aufgabenr}#1}\stepcounter{aufgabenr}}
		\stepcounter{aufgabenr}
\newcommand{\problem}[2][\mbox{}]
{\subsubsection*{}
\fbox{\parbox{\textwidth}{
\vspace{-16pt}
\Aufg[#1]
#2}}
\medskip}

% Umgebung "Lösung"
\newenvironment{lsg}{\par\bigskip\noindent{\emph{Solution.\ }}\footnotesize}{\hfill $\blacksquare$}


% Zahlenmengen
\newcommand{\Nat}{\ensuremath{\mathbb{N}}}
\newcommand{\Integer}{\ensuremath{\mathbb{Z}}}
\newcommand{\Rat}{\ensuremath{\mathbb{Q}}}
\newcommand{\Real}{\ensuremath{\mathbb{R}}}
\newcommand{\Comp}[1]{\ensuremath{\overline{#1}}}
\newcommand{\Conj}[1]{\ensuremath{\overline{#1}}}

\newcommand{\Lang}{\ensuremath{\mathcal{L}}}

\newcommand{\LA}{\ensuremath{\mathcal{L}_A}}
\newcommand{\M}{\ensuremath{\mathcal{M}}}
\newcommand{\TA}{\ensuremath{\operatorname{Th}(\Nat)}}
\newcommand{\PA}{\ensuremath{\mathsf{PA}}}

\newcommand{\Term}[1]{\ensuremath{\underline{#1}}}

\newcommand{\Pow}[1]{\ensuremath{\mathcal{P}(#1)}}




\begin{document}

\begin{center}
{\LARGE Homework 7 for \textbf{MATH 225B}}

%\vspace{0,5cm}

{\large Author: Ben Lickly \\ Due: Thursday March 19}
\end{center}


\problem{ 
Show that if $A$ is $\Sigma_3$, then there exists a computable $f$ such that
\begin{eqnarray*}
  x \in A & \iff & W_{f(x)} \text{ cofinite, } \\
  x \not\in A & \iff & W_{f(x)} \text{ Turing complete. }
\end{eqnarray*}
by modifying the construction for $\Sigma_3 \leq_m \operatorname{Cof}$ as indicated in class (coding $K$ into $W_{f(x)}$).
Give a detailed argument why this yields that $K \leq_T W_{f(x)}$ if $x \not \in A$.
}


\problem{
Show that there exists a set $A \leq_T \emptyset'$ such that $A$ is not Turing equivalent to a c.e.\ set. (Hence there exist non-c.e.\ $\Delta_2$-degrees.)
}

\begin{proof}
  We will construct the set $A$ in stages by \emph{finite extension}:
\begin{align*}
  & \chi_A = \bigcup_s f_s  \\
  f_s \le_T \emptyset', &\text{ for all finite strings of length $\ge s$} \\
\end{align*}
$\left\{f_s\right\}$ is a $\emptyset'$-computable sequence:
\[
f_s \subset f_{s+1} 
\]
%This implies $A \le_T \emptyset'$, since e.g. to decide $x \in A$ comp. $s$ s.t. $|f_s| \ge x$.

\subsection*{Requirements}
  We want to show that $A$ is not equivalent to any c.e. set.
  Since there are countably many c.e. sets, we can order them.
  Let $B_i$ be the $i$th c.e. set.
  Formally, we require:
\begin{align*}
  R_{e,i}: A \ne \Phi^{B_i}_e \\
\end{align*}

\subsection*{Construction}
We need to now construct the intermediate stages $f_s$ such that 
$
\chi_A = \bigcup_s f_s
$.

 Stage $0$: \[
 f_0 = \emptyset 
 \]

 Stage $s+1 = e+1$: Satisfy $R_{e}$

 Let $n = |f_s|$. (i.e. $A(n)$ is not yet fixed)

 Now use $\emptyset'$-oracle to test whether
 \begin{equation}
   (\exists \sigma)(\exists t)\ [\sigma \subset B_s \wedge \Phi^\sigma_{e,t}(n) \downarrow]
   \label{star}
 \end{equation}

 If (\ref{star}) holds: 
 let $\langle \sigma^*, t^* \rangle$ be least pair
 satisfying (\ref{star}) and set:
\begin{align*}
  f_{s+1} = f_s \frown \left(1 - \Phi^{\sigma^*}_{e,t^*}(n) \right) \\
\end{align*}
  hence $f_{s+1}(n) = A(n)$ is not the value computed by $\Phi^{\sigma^*}_e(n)$,
  and hence by use principle, not value computed by $\Phi^{B_s}_e(n)$.

  If (\ref{star}) does not hold, let:
  \begin{align*}
    f_{s+1} = f_s \frown 0 \\
  \end{align*}
  (no danger of having $\Phi^{B_s}_e(n) \downarrow$, so not action required)
\end{proof}


\problem[*]{
Show that for each $A \geq_T \emptyset'$, there exists a $B$ such that $B' \equiv_T A$.
}
 



\end{document}

