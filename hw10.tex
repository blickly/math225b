\documentclass[12pt]{article}

%\usepackage{a4wide}
\usepackage[centertags]{amsmath}
\usepackage{amsthm}
\usepackage{amssymb}
\usepackage{amsfonts}
\usepackage{amscd,amsbsy}
\usepackage{txfonts}
\usepackage[T1]{fontenc}

% \usepackage{eulervm}
\usepackage{eucal}
%\usepackage{textcomp}

%\usepackage{mathptmx}
\usepackage{enumerate}


\usepackage{pdfsync}


% Deutsche Texte
\usepackage[applemac]{inputenc}

% Seitenlayout
\pagestyle{empty}

\setlength{\textheight}{25cm}
\setlength{\textwidth}{16cm}

\setlength{\hoffset}{-1,5cm}
\setlength{\voffset}{-2,5cm}

\newcommand{\alabel}[1]{\mbox{(#1)}\hfill}
\newenvironment{alist}
        {\begin{list}{}%
                {\renewcommand{\makelabel}{\alabel}%
                \setlength{\parsep}{0.2cm}\setlength{\listparindent}{0cm}
                \setlength{\itemsep}{0.2cm}}}
        {\end{list}}

		\newcounter{aufgabenr}
\newcommand{\Aufg}[1][\mbox{}]{\subsubsection*{Problem \arabic{aufgabenr}#1}\stepcounter{aufgabenr}}
		\stepcounter{aufgabenr}
\newcommand{\problem}[2][\mbox{}]
{\subsubsection*{}
\fbox{\parbox{\textwidth}{
\vspace{-16pt}
\Aufg[#1]
#2}}
\medskip}

% Umgebung "Lösung"
\newenvironment{lsg}{\par\bigskip\noindent{\emph{Solution.\ }}\footnotesize}{\hfill $\blacksquare$}


% Zahlenmengen
\newcommand{\Nat}{\ensuremath{\mathbb{N}}}
\newcommand{\Integer}{\ensuremath{\mathbb{Z}}}
\newcommand{\Rat}{\ensuremath{\mathbb{Q}}}
\newcommand{\Real}{\ensuremath{\mathbb{R}}}
\newcommand{\Comp}[1]{\ensuremath{\overline{#1}}}
\newcommand{\Conj}[1]{\ensuremath{\overline{#1}}}

\newcommand{\Lang}{\ensuremath{\mathcal{L}}}

\newcommand{\LA}{\ensuremath{\mathcal{L}_A}}
\newcommand{\M}{\ensuremath{\mathcal{M}}}
\newcommand{\TA}{\ensuremath{\operatorname{Th}(\Nat)}}
\newcommand{\PA}{\ensuremath{\mathsf{PA}}}

\newcommand{\GN}[1]{\ensuremath{\ulcorner #1 \urcorner}}

\newcommand{\Term}[1]{\ensuremath{\underline{#1}}}

\newcommand{\Pow}[1]{\ensuremath{\mathcal{P}(#1)}}

\newcommand{\Ax}[1]{\ensuremath{\mathsf{#1}}}


\begin{document}

\begin{center}
{\LARGE Homework 10 for \textbf{MATH 225B}}

%\vspace{0,5cm}

{\large Author: Ben Lickly \\ Due: Thursday May 7}
\end{center}


\problem{
Show that $6 \to (3)^2_2$, but $5 \not\to (3)^2_2$.
}


\problem{
Show that $2^\Nat$ with the standard topology is a compact space and use this to give a topological proof of K\"onig's Lemma: If $T \subseteq 2^{<\Nat}$ is an infinite tree, then $[T] \neq \emptyset$.
}


\problem{
A group $G$ is \emph{free} if there exists a subset $X \subseteq G$ such that any element of $G$ can be written in one and only one way as a product of finitely many elements of $X$ and their inverses (disregarding trivial variations such as $st^{-1} = su^{-1}ut^{-1}$). In this case, we say $G$ is a free group on generators $X$. Think of a free group on generators $X$ as the set of all \emph{reduced} words over the symbol set $\{x, x^{-1}\colon x \in X\}$ (where a word is reduced if it does not contain a substring of the form $x^{-1}x$ or $xx^{-1}$). The group operation is concatenation and the identity is the empty string.

Show that if $G$ is a free group on generators $X$, then $X$ is a set of indiscernibles on $X$.
}


\problem{
Given a formula $\varphi(x,\vec{y})$, the \emph{least number principle} (LNP) for $\varphi$ is defined as
\[
	\forall \vec{y} (\exists x \: \varphi(x,\vec{y}) \; \to \; \exists z (\varphi(z,\vec{y}) \: \wedge \: \forall w < z \: \neg \varphi(w,\vec{y}))).
\]
Show that over $\PA^{-}$, LNP is equivalent to Induction
\[
	\forall \vec{y} (\varphi(0,\vec{y}) \: \wedge \: \forall x (\varphi(x,\vec{y}) \to \varphi(x+1,\vec{y})) \; \to \;  \forall x \: \varphi(x,\vec{y})).
\]
}


\problem{
Let $\M \models \PA$ be non-standard. A \emph{proper cut} in $\M$ is a set $I \subsetneq M$ that is an initial segment of $M$ and closed under successor, e.g.\ the standard model $\Nat$.
\begin{alist}

	\item[a] Show that if $\vec{a} \in M$ and $\M \models \varphi(b,\vec{a})$ for all $b \in I$, then there is $c > I$ in $M$ such that $\M \models \forall x \leq c \varphi(x,\vec{a})$.

	\item[b] Show that if $\M \models \PA^{-}$ is non-standard and satisfies the conclusion of (a), then $\M \models \PA$.
\end{alist}
}




\end{document}
